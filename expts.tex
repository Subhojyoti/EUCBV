In this section we conduct an extensive empirical evaluation of EUCBV against several other popular bandit algorithms.  We use cumulative regret as the metric of comparison. We implement the following algorithms:  KL-UCB\cite{garivier2011kl}, DMED\cite{honda2010asymptotically}, MOSS\cite{audibert2009minimax}, UCB1\cite{auer2002finite}, UCB-Improved\cite{auer2010ucb}, Median Elimination\cite{even2006action}, Thompson Sampling(TS)\cite{agrawal2011analysis}, OCUCB\cite{lattimore2015optimally}, Bayes-UCB(BU)\cite{kaufmann2012bayesian} and UCB-V\cite{audibert2009exploration}\footnote{The implementation for KL-UCB, Bayes-UCB and DMED were taken from \cite{CapGarKau12}}. The parameters of EUCBV algorithm for all the experiments are set as follows: $\psi=\frac{T}{K^2}$ and $\rho =0.5$ (as in Corollary \ref{Result:Corollary:2}).

\begin{figure}[!h]
    \centering
    \begin{tabular}{cc}
    \setlength{\tabcolsep}{0.1pt}
    \subfigure[0.25\textwidth][Expt-$1$: $20$ Bernoulli-distributed arms with $r_{i_{{i}\neq {*}}}=0.07$ and $r^{*}=0.1$]
    {
    		\pgfplotsset{
		tick label style={font=\Large},
		label style={font=\Large},
		legend style={font=\Large},
		ylabel style={yshift=32pt},
		%legend style={legendshift=32pt},
		}
        \begin{tikzpicture}[scale=0.5]
      	\begin{axis}[
		xlabel={timestep},
		ylabel={Cumulative Regret},
		grid=major,
        %clip mode=individual,grid,grid style={gray!30},
        clip=true,
        %clip mode=individual,grid,grid style={gray!30},
  		legend style={at={(0.5,1.5)},anchor=north, legend columns=3} ]
      	% UCB
		\addplot table{results/NewExpt/Expt1/UCBV01_comp_subsampled.txt};
		%\addplot table{results/NewExpt/Expt1/EclUCB01_1_comp_subsampled.txt};
		\addplot table{results/NewExpt/Expt1/EUCBV01_comp_subsampled.txt};
		\addplot table{results/NewExpt/Expt1/KLUCB01_comp_subsampled.txt};
		\addplot table{results/NewExpt/Expt1/MOSS01_comp_subsampled.txt};
		\addplot table{results/NewExpt/Expt1/DMED01_comp_subsampled.txt};
		\addplot table{results/NewExpt/Expt1/UCB01_comp_subsampled.txt};
		\addplot table{results/NewExpt/Expt1/TS02_comp_subsampled.txt};
		\addplot table{results/NewExpt/Expt1/OCUCB01_comp_subsampled.txt};
		\addplot table{results/NewExpt/Expt1/BU01_comp_subsampled.txt};
      	\legend{UCB-V,EUCBV,KL-UCB,MOSS,DMED,UCB1,TS,OCUCB,BU}      	
      	\end{axis}
      	\end{tikzpicture}
  		\label{fig:1}
    }
    &
    \subfigure[0.25\textwidth][Expt-$2$: $100$ Gaussian-distributed arms with $r_{i_{{i}\neq {*}:1-33}}=0.1$, $r_{i_{{i}\neq {*}:34-99}}=0.6$, $r^{*}_{i=100}=0.9$ and $\sigma_{i=1:100}^{2} = 0.3$]
    {
    		\pgfplotsset{
		tick label style={font=\Large},
		label style={font=\Large},
		legend style={font=\Large},
		}
        \begin{tikzpicture}[scale=0.5]
        \begin{axis}[
		xlabel={timestep},
		ylabel={Cumulative Regret},
        %clip mode=individual,grid,grid style={gray!30},
       	grid=major,
       	clip=true,
  		legend style={at={(0.5,1.5)},anchor=north, legend columns=3} ]
      	% UCB
        \addplot table{results/NewExpt/Expt2_2/UCB01_comp_subsampled.txt};
		%\addplot table{results/NewExpt/Expt2_2/clUCB01_comp_subsampled.txt};
		\addplot table{results/NewExpt/Expt2_2/EUCBV01_comp_subsampled.txt};
		\addplot table{results/NewExpt/Expt2_2/MOSS01_comp_subsampled.txt};
		\addplot table{results/NewExpt/Expt2_2/OCUCB01_comp_subsampled.txt};
		\addplot table{results/NewExpt/Expt2_2/MedElim_comp_subsampled.txt};
		\addplot table{results/NewExpt/Expt2_2/UCBR01_comp_subsampled.txt};
		%\addplot table{results/NewExpt/Expt2_2/EclUCB01_p_1_comp_subsampled.txt};
		%\legend{UCB1,ClusUCB,Med-Elim,MOSS,OCUCB,EClusUCB,UCB-Imp}
		\legend{UCB1,EUCBV,MOSS,OCUCB,Med-Elim,UCB-Imp}
      	\end{axis}
      	\end{tikzpicture}
   		\label{fig:2}
    }
    \end{tabular}
    \caption{Cumulative regret for various bandit algorithms on two stochastic K-armed bandit environments. }
    \label{fig:karmed}
    \vspace*{-1em}
\end{figure}
% For the purpose of performance comparison


\textbf{First experiment:} This experiment is conducted to observe the performance of EUCBV over a short horizon. The horizon $T$ is set to $60000$. These type of cases are frequently encountered in web-advertising domain. The testbed comprises of $20$ Bernoulli distributed arms with expected rewards of the arms as $r_{i_{{i}\neq {*}}}=0.07$ and $r^{*}=0.1$. The regret is averaged over $100$ independent runs and is shown in Figure \ref{fig:1}. EUCBV, MOSS, UCB1, UCB-V, KL-UCB, TS, BU and DMED are run in this experimental setup. Here not only we observe that EUCBV performs better than all the non-variance based  algorithms like MOSS, OCUCB, UCB-Improved and UCB1 but it also outperforms UCBV because of the choice of the exploration parameters. Because of the small gaps and short horizon $T$, we do not implement UCB-Improved and Median Elimination on this test-case. 

\textbf{Second experiment:} This experiment is conducted on a large horizon and over a large set of arms.    The horizon $T$ is set for a large duration of $2\times 10^{5}$. This testbed comprises of $100$ arms involving Gaussian reward distributions with expected rewards of the arms $r_{i_{{i}\neq {*}:1-33}}=0.1$, $r_{i_{{i}\neq {*}:34-99}}=0.6$, $r^{*}_{i=100}=0.9$ and variance set at $\sigma_{i}^{2} = 0.3,\forall i\in \A$. The regret is averaged over $100$ independent runs and is shown in Figure \ref{fig:2}. From the results in Figure \ref{fig:2}, we observe that EUCBV outperforms all the non-variance based algorithms MOSS, OCUCB, UCB1, UCB-Improved and Median-Elimination($\epsilon=0.1,\delta=0.1$). 
%Also the performance of UCB-Improved is poor in comparison to other algorithms, which is probably because of pulls wasted in initial exploration whereas EUCBV with the choice of $\psi$ and $\rho$ performs much better.

\begin{figure}[!h]
    \centering
    \begin{tabular}{cc}
    \subfigure[0.25\textwidth][Expt-$3$: $20$ to $100$ Bernoulli-distributed arms with $r_{i_{{i}\neq {*}}}=0.05$ and $r^{*}=0.1$]
    {
    	\pgfplotsset{
		tick label style={font=\Large},
		label style={font=\Large},
		legend style={font=\Large},
		ylabel style={yshift=32pt},
		}
        \begin{tikzpicture}[scale=0.5]
        \begin{axis}[
		xlabel={Arms},
		ylabel={Cumulative Regret},
        %clip mode=individual,grid,grid style={gray!30},
		grid=major,
		clip=true,
  		legend style={at={(0.5,1.3)},anchor=north, legend columns=3} ]
        % UCB
		\addplot table{results/NewExpt/Expt3/plotFinalMOSS20_100.txt};
		\addplot table{results/NewExpt/Expt3/plotFinalEUCBV20_100.txt};
		\addplot table{results/NewExpt/Expt3/plotFinalOCUCB20_100.txt};
      	\legend{MOSS,EUCBV,OCUCB}
      	\end{axis}
        \end{tikzpicture}
        \label{fig:3}
    }
%    &
%    \subfigure[0.32\textwidth][Expt-$4$: Different cluster experiment]
%    {
%    		\pgfplotsset{
%		tick label style={font=\Huge},
%		label style={font=\Huge},
%		legend style={font=\Large},
%		ylabel style={yshift=32pt},
%		}
%        \begin{tikzpicture}[scale=0.35]
%      	\begin{axis}[
%		ylabel={Cumulative Regret},
%		xlabel={Clusters},
%		grid=major,
%        %clip mode=individual,grid,grid style={gray!30},
%        clip=true,
%        %clip mode=individual,grid,grid style={gray!30},
%  		legend style={at={(0.5,1.3)},anchor=north, legend columns=3} ]
%      	% UCB
%		%\addplot table{results/NewExpt/Expt4/plotFinalAclUCB012.txt};
%		\addplot table{results/NewExpt/Expt4/plotFinalEclUCB012.txt};
%		%\addplot table{results/NewExpt/Expt4/plotFinalEclUCB_AE012.txt};
%      	%\legend{EClusUCBA,AClusUCBA,EClusUCB,AClusUCB} 
%      	\legend{EClusUCB} 
%      	\end{axis}
%      	\draw [red,thick] (.61,5.25) circle (0.3 cm);
%      	code={
%        \node[black,above] at (axis cs:3.0,4.27){\tiny{EClusUCB-AE}};
%    		}
%      	\end{tikzpicture}
%  		\label{fig:4}
%    }
	\end{tabular}
	\label{fig:furtherExpt1}
    \caption{Further Experiments with EUCBV}
    \vspace*{-1em}
\end{figure}
%\vspace*{-0.5em}
\textbf{Third experiment:} This experiment is conducted to show the stability and performance of EUCBV over a very large horizon and over a large number of arms. This testbed comprises of $20-100$ (interval of $10$) arms with Bernoulli reward distributions, where the expected rewards of the arms are $r_{i_{{i}\neq {*}}}=0.05$ and $r^{*}=0.1$. For each of these testbeds of $20-100$ arms, we report the cumulative regret averaged over $100$ independent runs. The horizon is set at $T=10^{5} + K_{20:100}^{3}$ timesteps. Please note that algorithms like Thompson Sampling or Bayes-UCB are too slow to be run for such large arms (see \citet{lattimore2015optimally}) and  over such large horizon. We report the performance of MOSS, OCUCB and EUCBV only over this uniform gap setup. From the results in Figure \ref{fig:3}, it is evident that the growth of regret for EUCBV  is much lower than that of OCUCB and MOSS. This also corroborates the finding of \citet{lattimore2015optimally} which states that MOSS breaks down only when the number of arms are exceptionally large or the horizon is unreasonably high and gaps are very small. We consistently see that in uniform gap testcases EUCBV outperforms OCUCB.


