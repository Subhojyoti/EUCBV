\subsection{Lemma 1}

A technical lemma used to prove Theorem \ref{Result:Theorem:1} is presented below.

\begin{lemma}
\label{results:Lemma:1}
If $T\geq K^{2.7}$, $\psi=\dfrac{T}{ K^2}$, $\rho=\dfrac{1}{2}$ and $m\leq \dfrac{1}{2} \log_2(\dfrac{T}{e}) $, then,
\begin{align*}
\dfrac{\rho m \log(2)}{\log(\psi T) - 2m\log( 2)} \leq 1
\end{align*}
\end{lemma}

\begin{proof}
The proof is given in Section \ref{sec:proofTheorem:Lemma1}.
\end{proof}

We present below the main theorem of the paper which establishes the regret upper bound for the EUCBV  algorithm. 

\subsection{Main Theorem}
\begin{theorem}
\label{Result:Theorem:1}
For $T\geq K^{2.7}$, the regret $R_T$ for EUCBV satisfies
\begin{align*}
 \E [R_{T}] \leq &\sum\limits_{i\in \A :\Delta_{i} > b}\bigg\lbrace\bigg(\dfrac{C_{1}(\rho)T^{1-\rho}}{\Delta_{i}^{2\rho -1}}\bigg) + \bigg(\Delta_{i}+\dfrac{41\log{(\psi  T\Delta_{i}^{4})}}{\Delta_{i}}\bigg) + \bigg(\dfrac{ C_{2}(\rho) T^{1-\rho}}{\Delta_{i}^{2\rho-1}} \bigg)\bigg \rbrace \\ 
 %%%%%%%%%%%%%%%%
  & +\sum\limits_{i\in \A :0 < \Delta_{i}\leq b}\bigg(\dfrac{C_{2}(\rho)T^{1-\rho}}{b^{2\rho -1}} \bigg) + \max_{i\in \A :0 < \Delta_{i}\leq b}\Delta_{i}T
\end{align*}
for all $b\geq\sqrt{\frac{e}{T}}$. In the above, $C_1(x) = \frac{2^{2+x}.9^{x}}{\psi^{x}}$ and $C_2(x) = \frac{2^{\frac{\rho}{2}+\frac{9}{4}}.3^{x+\frac{1}{2}}}{\psi^{x}}$.
\end{theorem}

\begin{proof}
The proof comprises of three modules. In the first module we prove the necessary conditions for arm elimination within a specified number of rounds, which is motivated by the technique in \cite{auer2010ucb}. In this module we combine the approach of \cite{audibert2009exploration} with that of  \cite{auer2010ucb} while breaking down the confidence interval term which contains the estimated variance   term for an arm. Please note that even though \cite{audibert2009exploration} uses Bernstein inequality to obtain the  bound, we use Chernoff-Hoeffding bound. This is because of our choice of $\rho$ which cannot be decreased below $\frac{1}{2}$ as it may lead to a regret polynomial in $T$ and usage of Bernstein inequality will force the $\rho$ to take a value lower than $\frac{1}{2}$. The second module bounds the number of pulls required if an arm is eliminated on or before a particular number of rounds. Note that the number of pulls allocated in a round $m$ for each arm is atmost $n_{m}:=\bigg\lceil\frac{\log{(\psi T\epsilon_{m}^{2})}}{2\epsilon_{m}}\bigg\rceil$ which is much lower than the pulls of each arm required by UCB-Improved or Median-Elimination. The third module deals with bounding the regret given a sub-optimal arm eliminates the optimal arm. The detailed proof is given in Section \ref{sec:proofTheorem:Theorem1}.
\end{proof}
Next, we specialize the result of Theorem \ref{Result:Theorem:1} in Corollary \ref{Result:Corollary:1} and Corollary \ref{Result:Corollary:2}.

\subsection{Corollary 1}
\begin{corollary}[\textbf{\textit{Gap-dependent bound}}]
\label{Result:Corollary:1}
With $\psi=\frac{T}{K^2}$ and $\rho=\frac{1}{2}$, we have the following gap-dependent bound for the regret of EUCBV:
\begin{align*}
\E [R_T] & \sum_{i\in \A:\Delta_{i} > b}\bigg\lbrace 34 K + \dfrac{82\log{(\dfrac{T\Delta_{i}^{2}}{ K})}}{\Delta_{i}}\bigg\rbrace 
 + \max\limits_{i\in \A:\Delta_{i}\leq b}\Delta_{i}T 
	\end{align*} 
\end{corollary}
\begin{proof}
The proof is given in Section \ref{sec:proofTheorem:Corollary1}.
\end{proof}
Thus, we clearly see that the most significant term in the gap-dependent bound is $\dfrac{82K\log{(T\Delta^{2}/K)}}{\Delta}$ and it is better than UCB1, UCBV, MOSS and UCB-Improved. In \citet{audibert2010best} the authors define the term $H_1=\sum_{i=1}^{K}\frac{1}{\Delta_i^2}$ as the hardness of a problem and in \citet{bubeck2012regret} the authors conjectured that the gap-dependent regret upper bound can match the quantity of $O\left(\dfrac{K\log{(T/H_1)}}{\Delta}\right)$. But \citet{lattimore2015optimally} proved that the gap-dependent regret bound cannot be lower than $O\left(\sum_{i=2}^{K}\frac{\log\left(T/H_i\right)}{\Delta_i}\right)$, where $H_i=\sum_{j=1}^{K}\min\lbrace \frac{1}{\Delta_i^2},\frac{1}{\Delta_j^2}\rbrace$ and only in the worst case scenario, when all the gaps are equal then $H_1=H_{i}=\sum_{i=1}^{K}\frac{1}{\Delta^2}$. In such a scenario the EUCBV gap-dependent bound of $O\left(\dfrac{K\log{(T\Delta^{2}/ K)}}{\Delta}\right)$ reduces to $O\left(\dfrac{K\log{(T/H_1)}}{\Delta}\right)$ and hence matches the gap-dependent bound of OCUCB.

\subsection{Corollary 2}

\begin{corollary}[\textbf{\textit{Gap-independent bound}}]
\label{Result:Corollary:2}
With $\psi=\frac{T}{K^2}$ and $\rho=\frac{1}{2}$, we have the following gap-independent bound for the regret of EUCBV:
\begin{align*}
\E[R_{T}] & \leq 51 K^2 + 82\sqrt{KT}
	\end{align*} 
\end{corollary}
\begin{proof}
The proof is given in Section \ref{sec:proofTheorem:Corollary2}.
\end{proof}
Here, in the gap-independent bound of EUCBV the most significant term is $O\left(\sqrt{KT}\right)$ which exactly matches the upper bound of MOSS and OCUCB and is better than UCB-Improved, UCB1 and UCBV.