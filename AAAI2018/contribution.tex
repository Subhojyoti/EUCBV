In this paper we propose the Efficient-UCB-Variance (henceforth referred to as EUCBV) algorithm for the stochastic MAB setting. EUCBV combines the approach of UCB-Improved, CCB \citep{liu2016modification} and UCBV algorithms. EUCBV, by virtue of taking into account the empirical variance of the arms, exploration parameters  and non-uniform arm selection (as opposed to UCB-Improved), performs significantly better than the existing algorithms in the stochastic MAB setting. EUCBV outperforms UCBV \citep{audibert2009exploration} which also takes into account empirical variance but is less powerful than EUCBV because of the usage of exploration regulatory factor by EUCBV. Also, we carefully design the confidence interval term with the variance estimates along with the pulls allocated to each arm to balance the risk of eliminating the optimal arm against excessive optimism. Theoretically we refine the analysis of \citet{auer2010ucb} and prove that for $T\geq K^{2.4}$ our algorithm is order optimal and achieves a worst case gap-independent regret bound of $O\left( \sqrt{KT} \right)$ which is same as that of MOSS and OCUCB but better than that of UCBV, UCB1 and UCB-Improved. Also, the gap-dependent regret bound of EUCBV is better than UCB1, UCB-Improved and MOSS but is poorer than OCUCB. However, EUCBV's gap-dependent bound matches OCUCB in the worst case scenario when all the gaps are equal. Through our theoretical analysis we establish the exact values of the exploration parameters for the best performance of EUCBV. Our proof technique is highly generic and can be easily extended to other MAB settings. An illustrative table containing the bounds is provided in Table \ref{tab:comp-bds}. 


\begin{table}[t]
\caption{Regret upper bound of different algorithms}
\label{tab:comp-bds}
\begin{center}
\begin{tabular}{p{3em}p{9em}p{7em}}
\toprule
Algorithm  &   \hspace*{1mm}Gap-Dependent & Gap-Independent \\
\hline
EUCBV		& $O\left( \dfrac{K\sigma_{\max}^{2}\log (\frac{T\Delta^2}{K})}{\Delta}\right)$ & $O\left(\sqrt{KT}\right)$\\
UCB1        & $O\left( \dfrac{K\log T}{\Delta} \right)$ & $O\left(\sqrt{KT\log T}\right)$ \\%\midrule
UCBV        & $O\left( \dfrac{K\sigma_{\max}^{2}\log T}{\Delta} \right)$ & $O\left(\sqrt{KT\log T}\right)$ \\
UCB-Imp 		& $O\left( \dfrac{K\log (T\Delta^2)}{\Delta} \right)$ & $O\left(\sqrt{KT\log K}\right)$ \\%\midrule
MOSS	     	& $O\left( \dfrac{K^2\log (T\Delta^2 /K)}{\Delta}\right)$ & $O\left(\sqrt{KT}\right)$\\%\midrule
OCUCB     	& $O\left( \dfrac{K\log (T/ H_{i})}{\Delta}\right)$ & $O\left(\sqrt{KT}\right)$\\\midrule
\end{tabular}
\end{center}
\vspace*{-2em}
\end{table}


Empirically, we show that EUCBV, owing to its estimating the variance of the arms, exploration parameters and non-uniform arm pull, performs significantly better than MOSS, OCUCB, UCB-Improved, UCB1, UCBV, TS, BU, DMED, KLUCB and Median Elimination algorithms. Note that except UCBV, TS, KLUCB and BU (the last three with Gaussian priors) all the aforementioned algorithms do not take into account the empirical variance estimates of the arms. Also, for the optimal performance of TS, KLUCB and BU one has to have the prior knowledge of the type of distribution, but EUCBV requires no such prior knowledge. EUCBV is the first arm-elimination algorithm that takes into account the variance estimates of the arm for minimizing cumulative regret and thereby answers an open question raised by \citet{auer2010ucb}, where the authors conjectured that an UCB-Improved like arm-elimination algorithm can greatly benefit by taking into consideration the variance of the arms. Also, it is the first algorithm that follows the same proof technique of UCB-Improved and achieves a gap-independent regret bound of $O\left( \sqrt{KT} \right)$ thereby, closing the gap of UCB-Improved which achieved a gap-independent regret bound of $O\left( \sqrt{KT\log K} \right)$. 
	
	The rest of the paper is organized as follows. In section~\ref{sec:eucbv} we present the  EUCBV algorithm. Our main theoretical results are stated in section~\ref{sec:results}, while the proofs are established in   section \ref{sec:proofTheorem}. Section~\ref{sec:expt} contains results and discussions from our numerical experiments. We draw our conclusions in section \ref{sec:conc} and section \ref{sec:app} is Appendix (supplementary material).
	
	%discuss about future works. 
	
	%The section \ref{sec:app} containing further proofs is given as supplementary.
	
	
	