
\subsection{Proof of Lemma 1} \label{App:Lemma:1}

\begin{lemma}
%\label{proofTheorem:Lemma:1}
If $T\geq K^{2.4}$, $\psi=\dfrac{T}{ K^2}$, $\rho=\dfrac{1}{2}$ and $m\leq \dfrac{1}{2} \log_2\left(\dfrac{T}{e}\right) $, then,
\begin{align*}
\dfrac{\rho m \log(2)}{\log(\psi T) - 2m\log( 2)} \leq \frac{3}{2}
\end{align*}
\end{lemma}

\begin{proof}
The proof is based on contradiction. Suppose
\begin{eqnarray*}
\dfrac{\rho m \log(2)}{\log(\psi T) - 2m\log( 2)} > \frac{3}{2}.
\end{eqnarray*}
Then, with $\psi=\dfrac{T}{ K^2}$ and $\rho=\dfrac{1}{2}$, we obtain
\begin{eqnarray*}
6\log(K) 
&>& 6\log(T) - 7m\log(2) \\
&\overset{(a)}{\ge}& 6\log(T) - \frac{7}{2} \log_2\left(\frac{T}{e}\right) \log(2) \\
&=& 2.5\log(T) + 3.5 \log_2(e)\log(2)  \\
&\overset{(b)}{=}& 2.5\log(T) +3.5
\end{eqnarray*}
where $(a)$ is obtained using $m\leq \dfrac{1}{2} \log_2\left(\dfrac{T}{e}\right)$, while $(b)$ follows from the identity $\log_2(e)\log(2) =1$. Finally, for $T\ge K^{2.4}$ we obtain, $6\log(K)>6\log(K)+3.5$, which is a contradiction.
\hfill $\blacksquare$	
\end{proof}

\subsection{Proof of Lemma 2}
\label{App:Lemma:2}
\begin{lemma}
%\label{proofTheorem:Lemma:2}
If $T\geq K^{2.4}$, $\psi=\dfrac{T}{ K^2}$, $\rho =\dfrac{1}{2}$, $m_i = min\lbrace m|\sqrt{4\epsilon_{m} } < \dfrac{\Delta_i}{4} \rbrace $ and $c_{i} =\sqrt{\frac{\rho (\hat{v}_i + 2)\log (\psi T\epsilon_{m_{i}})}{4 z_i}}$, then, $c_{i} < \dfrac{\Delta_i}{4}$.
\end{lemma}

\begin{proof}

	In the $m_i$-th round since $z_i\geq n_{m_i}$, by substituting $z_i$ with $n_{m_i}$ we can show that, 

\begin{align*}
	c_{i} &\leq \sqrt{\dfrac{\rho (\hat{v}_i + 2)\epsilon_{m_{i}}\log (\psi T\epsilon_{m_{i}})}{2\log(\psi T\epsilon_{m_{i}}^{2})}} \overset{(a)}{\leq} \sqrt{\dfrac{2\rho\epsilon_{m_{i}}\log (\frac{\psi T\epsilon_{m_{i}}^{2}}{\epsilon_{m_{i}}})}{\log(\psi T\epsilon_{m_{i}}^{2})}} \\
	%%%%%%%%%%%%%%%%%%%%%%%%%%
	& = \sqrt{\dfrac{2\rho\epsilon_{m_{i}}\log (\psi T\epsilon_{m_{i}}^{2}) - 2\rho\epsilon_{m_{i}}\log (\epsilon_{m_{i}})}{\log(\psi T\epsilon_{m_{i}}^{2})}} \\
	%%%%%%%%%%%%%%%%%%%%%%%%%%
	& \leq  \sqrt{2\rho\epsilon_{m_{i}} - \dfrac{2\rho\epsilon_{m_i}\log(\frac{1}{2^{m_i}})}{\log(\psi T \frac{1}{2^{2m_i}})}} \\
	%%%%%%%%%%%%%%%%%%%%%%%%%%
	&\leq \sqrt{2\rho\epsilon_{m_{i}} + \dfrac{2\rho\epsilon_{m_i}\log(2^{m_i})}{\log(\psi T) - \log( 2^{2m_i})}}\\
	%%%%%%%%%%%%%%%%%%%%%%%%%%
	& \leq \sqrt{2\rho\epsilon_{m_{i}} + \dfrac{2\rho\epsilon_{m_i}m_i \log(2)}{\log(\psi T) - 2m_i\log( 2)}} \\ 
	%%%%%%%%%%%%%%%%%%%%%%%%%%
	 & \overset{(b)}{\leq} \sqrt{2\rho\epsilon_{m_{i}} + 2.\frac{3}{2}\epsilon_{m_i}} 
	  < \sqrt{4\epsilon_{m_i}} < \dfrac{\Delta_{i}}{4}
	\end{align*}
In the above simplification, $(a)$ is due to $\hat{v}_i \in [0,1]$, while $(b)$ is obtained using Lemma~\ref{proofTheorem:Lemma:1}.
% < \dfrac{\Delta_{i}}{4 \sigma_i^2} \overset{(c)}{<}
% and $(c)$ happens because $\sigma_i \in (0,1]$. 
%Similarly, it can be shown that $c^* < \frac{\Delta_i}{4}$ in round $m_i$.
\hfill $\blacksquare$	
\end{proof}


\subsection{Proof of Lemma 3}
\label{App:Lemma:3}
\begin{lemma}
If $m_i = min\lbrace m|\sqrt{4\epsilon_{m} } < \dfrac{\Delta_i}{4} \rbrace $,  $\bar{c}_i=\sqrt{\dfrac{\rho (\sigma_{i}^{2}+\sqrt{\epsilon_{m_{i}}} + 2)\log(\psi T\epsilon_{m_{i}})}{4z_i}}$ and $n_{m_i} = \frac{\log{(\psi T\epsilon_{m_{i}})}}{2\epsilon_{m_{i}}}$ then we can show that,
\begin{align*}
\mathbb{P}\left( \hat{r}_{i} > r_{i}+ \bar{c}_i\right) 
+ \mathbb{P}\left( \hat{v}_{i}\geq \sigma_{i}^{2}+\sqrt{\epsilon_{m_{i}}}\right) \leq \dfrac{2}{(\psi  T\epsilon_{m_{i}})^{\frac{3\rho}{2}}}
\end{align*}
\end{lemma}

\begin{proof}

We start by recalling that,

\begin{align}
\mathbb{P}(\hat{r}_{i}> r_{i} + c_{i})
&\leq \mathbb{P}\left( \hat{r}_{i} > r_{i}+ \bar{c}_i\right) 
+ \mathbb{P}\left( \hat{v}_{i}\geq \sigma_{i}^{2}+\sqrt{\epsilon_{m_{i}}}\right)\label{eq:prob_eq3}
\end{align}
where 
\begin{align*}
&c_i =\sqrt{\frac{\rho (\hat{v}_i + 2)\log (\psi T\epsilon_{m_{i}})}{4 z_i}} \\
&\bar{c}_i=\sqrt{\dfrac{\rho (\sigma_{i}^{2}+\sqrt{\epsilon_{m_{i}}} + 2)\log(\psi T\epsilon_{m_{i}})}{4 z_i}}.
\end{align*}

Note that, substituting $ z_i \geq n_{m_i} \geq \frac{\log{(\psi T\epsilon_{m_{i}})}}{2\epsilon_{m_{i}}}$, $\bar{c}_i$ can be simplified to obtain,
\begin{align}
\bar{c}_i
\leq \sqrt{\dfrac{\rho\epsilon_{m_{i}}(\sigma_{i}^{2}+\sqrt{\epsilon_{m_{i}}} + 2)}{2}}\leq \sqrt{ \epsilon_{m_{i}}}.
\label{si_bar_equn}
\end{align}
%
The first term in the LHS of (\ref{eq:prob_eq3}) can be bounded using the Bernstein inequality as below:
\begin{align}
&\mathbb{P}\left( \hat{r}_{i} > r_{i}+ \bar{c}_i\right)\nonumber 
\le \exp\left(- \dfrac{(\bar{c}_i)^2 z_{i}}{2\sigma_i^2 + \frac{2}{3}\bar{c}_i} \right)\nonumber 
%%%%%%%%%%%%%%%
\\
& \overset{(a)}{\le} \exp\left(- \rho \left(\dfrac{3\sigma_{i}^{2}+3\sqrt{\epsilon_{m_{i}}} + 6}{6\sigma_i^2 + 2\sqrt{\epsilon_{m_i}}} \right)\log(\psi  T\epsilon_{m_{i}}\right)\nonumber \\
%%%%%%%%%%%%%%%
% &\le \exp\left(- \rho (\sigma_{i}^{2}+\sqrt{\epsilon_{m_{i}}} + 2)\log(\psi  T\epsilon_{m_{i}})\right)\nonumber \\
%%%%%%%%%%%%%%%
& \overset{(b)}{\leq} \exp\left(- \rho \log(\psi  T\epsilon_{m_{i}})\right) 
%%%%%%%%%%%%%%%
\le \dfrac{1}{(\psi  T\epsilon_{m_{i}})^{\frac{3\rho}{2}}}
\label{lhs1_equn}
\end{align}
where, $(a)$ is obtained by substituting equation \ref{si_bar_equn} and $(b)$ occurs because for all $\sigma_{i}^2 \in [0,\frac{1}{4}]$, $\left(\frac{3\sigma_{i}^{2}+3\sqrt{\epsilon_{m_{i}}} + 6}{6\sigma_i^2 + 2\sqrt{\epsilon_{m_i}}}\right) \geq \frac{3}{2}$ .

 
The second term in the LHS of (\ref{eq:prob_eq2}) can be simplified as follows:
\begin{align}
&\mathbb{P}\bigg\lbrace \hat{v}_{i}\geq \sigma_{i}^{2}+\sqrt{\epsilon_{m_{i}}}\bigg\rbrace\nonumber\\
%%%%%%%%%%%%%%%%%%
&\leq \mathbb{P}\bigg\lbrace \dfrac{1}{n_{i}}\sum_{t=1}^{n_{i}}(X_{i,t}-r_{i})^{2}-(\hat{r}_{i}-r_{i})^{2}\geq \sigma_{i}^{2}+\sqrt{\epsilon_{m_{i}}}\bigg\rbrace\nonumber\\
%%%%%%%%%%%%%%%%%%
&\leq \mathbb{P}\bigg\lbrace \dfrac{\sum_{t=1}^{n_{i}}(X_{i,t}-r_{i})^{2}}{n_{i}}\geq \sigma_{i}^{2}+\sqrt{\epsilon_{m_{i}}} \bigg\rbrace\nonumber\\
%%%%%%%%%%%%%%%%%%
&\overset{(a)}{\leq} \mathbb{P}\bigg\lbrace \dfrac{\sum_{t=1}^{n_{i}}(X_{i,t}-r_{i})^{2}}{n_{i}}\geq \sigma_{i}^{2} + \bar{c}_i\bigg\rbrace \nonumber\\
%%%%%%%%%%%%%%%%%%
&\overset{(b)}{\leq} \exp\left(- \rho \left(\dfrac{3\sigma_{i}^{2}+3\sqrt{\epsilon_{m_{i}}} + 6}{6\sigma_i^2 + 2\sqrt{\epsilon_{m_i}}} \right)\log(\psi  T\epsilon_{m_{i}})\right)
%%%%%%%%%%%%%%%%%
\le \dfrac{1}{(\psi  T\epsilon_{m_{i}})^{\frac{3\rho}{2}}}
\label{lhs2_equn}
\end{align}
where inequality $(a)$ is obtained using (\ref{si_bar_equn}), while $(b)$ follows from the Bernstein inequality. 

Thus, using (\ref{lhs1_equn}) and (\ref{lhs2_equn}) in (\ref{eq:prob_eq3}) we obtain $\mathbb{P}(\hat{r}_{i}> r_{i} + c_{i})\le \dfrac{2}{(\psi  T\epsilon_{m_{i}})^{\frac{3\rho}{2}}}$.
\hfill $\blacksquare$	
\end{proof}


\subsection{Proof of Lemma 4}
\label{App:Lemma:4}
\begin{lemma}
%\label{proofTheorem:Lemma:4}
If $m_i = min\lbrace m|\sqrt{4\epsilon_{m} } < \dfrac{\Delta_i}{4} \rbrace $, $c_{i} =\sqrt{\dfrac{\rho(\hat{v}_i + 2)\log (\psi T\epsilon_{m_{i}})}{4 z_{i}}}$ and $n_{m_i}=\dfrac{\log{(\psi T\epsilon_{m_{i}}^{2})}}{2\epsilon_{m_{i}}}$ then in the $m_i$-th round, 
\begin{align*}
\Pb\lbrace c^{*} > c_i \rbrace \leq \dfrac{32K\log T}{(\psi T)^{3\rho}}\sum_{m=0}^{m_i}\dfrac{1}{\epsilon_{m_i}^{3\rho + 1}}. 
\end{align*}
\end{lemma}

\begin{proof}
From the definition of $c_i$ we know that $c_i\propto \frac{1}{z_i}$ as $\psi$ and $T$ are constants. Therefore in the $m_i$-th round,
\begin{align*}
&\Pb\lbrace c^{*} > c_i \rbrace
%%%%%%%%%%%%%%%%%%%%%%%%%%%%%%%%%%%%%
\leq  \Pb\lbrace  z^* < z_i  \rbrace \\
%%%%%%%%%%%%%%%%%%%%%%%%%%%%%%%%%%%%%
&\leq \sum_{m=0}^{m_i}\sum_{z^* =1}^{n_{m_i}}\sum_{z_i =1}^{n_{m_i}}\bigg(\Pb\lbrace \hat{r}^* < r^* - c^{*}\rbrace + \Pb\lbrace \hat{r}_i > r_i + c_i\rbrace\bigg)
\end{align*}

Again, we can show that
\begin{align*}
&\sum_{m=0}^{m_i}\sum_{z^* =1}^{n_{m_i}}\sum_{z_i =1}^{n_{m_i}}\mathbb{I}_{i}\cup\mathbb{I}_{*} \leq \sum_{m=0}^{m_i}|B_{m_i}|n_{m_i} \\
%%%%%%%%%%%%%%%%%%%%%%%%%%%%%%%%%%%%
&\leq \sum_{m=0}^{m_i}\dfrac{4K}{(\psi T \epsilon_{m_i})^{\frac{3\rho}{2}}}\dfrac{\log{(\psi T\epsilon_{m_{i}}^{2})}}{2\epsilon_{m_{i}}} 
%\leq 
%%%%%%%%%%%%%%%%%%%%%%%%%%%%%%%%%%%%%
\leq \sum_{m=0}^{m_i}\dfrac{4K}{(\psi T \epsilon_{m_i})^{\frac{3\rho}{2}}}\dfrac{2\log(\frac{T}{K})}{\epsilon_{m_i}}
\end{align*}

Now, applying Bernstein inequality and following the same way as in Lemma \ref{proofTheorem:Lemma:3} we can show that,
\begin{align*}
&\Pb\lbrace \hat{r}^* < r^* - c^{*}\rbrace \leq \exp(- \frac{(c^{*})^2}{2\sigma_*^2 + \frac{2 c^{*}}{3}} z^*)\leq \frac{4}{(\psi T\epsilon_{m_i})^{\frac{3\rho}{2}}} \\ 
%%%%%%%%%%%%%%%%%%%%%%%%%%%%%%%%%%%
&\Pb\lbrace \hat{r}_i > r_i + c_i\rbrace \leq \exp(- \frac{(c_{i})^2}{2\sigma_i^2 + \frac{2 c_{i}}{3}} z_i)\leq \frac{4}{(\psi T\epsilon_{m_i})^{\frac{3\rho}{2}}}
\end{align*}

Hence, summing everything up, 
\begin{align*}
&\Pb\lbrace c^{*} > c_i \rbrace \\
%%%%%%%%%%%%%%%%%%%%%%%%%%%%%%%%
&\leq \sum_{m=0}^{m_i}\sum_{z^* =1}^{n_{m_i}}\sum_{z_i =1}^{n_{m_i}}\bigg(\Pb\lbrace \hat{r}^* < r^* - c^{*}\rbrace + \Pb\lbrace \hat{r}_i > r_i + c_i\rbrace\bigg)\\
%%%%%%%%%%%%%%%%%%%%%%%%%%%%%%%%
&\leq \sum_{m=0}^{m_i}\dfrac{32K\log T}{(\psi T\epsilon_{m_i})^{3\rho}\epsilon_{m_i}} \leq 
\dfrac{32K\log T}{(\psi T)^{3\rho}}\sum_{m=0}^{m_i}\dfrac{1}{\epsilon_{m_i}^{3\rho + 1}} 
\end{align*}
\hfill $\blacksquare$	
\end{proof}


\subsection{Proof of Lemma 5}
\label{App:Lemma:5}
\begin{lemma}
%\label{proofTheorem:Lemma:5}
If $m_i = min\lbrace m|\sqrt{4\epsilon_{m} } < \dfrac{\Delta_i}{4} \rbrace $, $c_{i} =\sqrt{\frac{\rho (\hat{v}_i + 2)\log (\psi T\epsilon_{m_{i}})}{4 z_i}}$ and $n_{m_i}=\dfrac{\log{(\psi T\epsilon_{m_{i}}^{2})}}{2\epsilon_{m_{i}}}$ then in the $m_i$-th round, 
\begin{align*}
\Pb\lbrace z_i < n_{m_i} \rbrace \leq \dfrac{32K\log T}{(\psi T)^{3\rho}}\sum_{m=0}^{m_i}\dfrac{1}{\epsilon_{m_i}^{3\rho + 1}}.
\end{align*}
\end{lemma}

\begin{proof}
Following a similar argument as in Lemma \ref{proofTheorem:Lemma:4}, we can show that in the $m_i$-th round,
\begin{align*}
\Pb\lbrace z_i < n_{m_i} \rbrace &\leq \sum_{m=0}^{m_i}\sum_{z_i =1}^{n_{m_i}}\sum_{z^* =1}^{n_{m_i}}\bigg(\Pb\lbrace \hat{r}^* > r^* - c^{*}\rbrace + \Pb\lbrace \hat{r}_i < r_i + c_i\rbrace\bigg) \\
%%%%%%%%%%%%%%%%%%%%%%%%%%%%%%%%%%%%
&\leq \dfrac{32K\log T}{(\psi T)^{3\rho}}\sum_{m=0}^{m_i}\dfrac{1}{\epsilon_{m_i}^{3\rho + 1}}.
\end{align*}
\hfill $\blacksquare$	
\end{proof}

\subsection{Proof of Lemma 6}
\label{App:Lemma:6}
\begin{lemma}
%\label{proofTheorem:Lemma:5}
For $T\geq K^{2.4}$, $\epsilon_{m_i}\geq \sqrt{\frac{e}{T}}$, $\psi=\frac{T}{K^2}$ and $\rho=\frac{1}{2}$,  
\begin{align*}
\dfrac{6K}{(\psi T \epsilon_{m_i})^{\frac{3\rho}{2}}} > \dfrac{K\log T}{(\psi T)^{3\rho}}\sum_{m=0}^{m_i}\dfrac{1}{\epsilon_{m_i}^{3\rho + 1}}
\end{align*}
\end{lemma}

\begin{proof}
We prove this lemma by contradiction. Suppose,

\begin{align}
\dfrac{6K}{(\psi T \epsilon_{m_i})^{\frac{3\rho}{2}}} < \dfrac{K\log T}{(\psi T)^{3\rho}}\sum_{m=0}^{m_i}\dfrac{1}{\epsilon_{m_i}^{3\rho + 1}} \label{eq:contra:1}
\end{align}

Again, we know that 
\begin{align*}
\dfrac{6K}{(\psi T \epsilon_{m_i})^{\frac{3\rho}{2}}} \overset{(a)}{\leq} \dfrac{6K}{( \frac{T^2}{K^2} \epsilon_{m_i})^{\frac{3}{4}}} \overset{(b)}{\leq} \dfrac{6K}{( \frac{T^2}{K^2} \sqrt{\frac{e}{T}})^{\frac{3}{4}}}
< \dfrac{6K^{\frac{5}{2}}}{T^{\frac{9}{8}}}
\end{align*}

where, in $(a)$ we substitute the values of $\rho$ and $\psi$ and $(b)$ happens because $\epsilon_{m_i} \geq \sqrt{\frac{e}{T}}$. Similarly we can show that,

\begin{align*}
\dfrac{K\log T}{(\psi T)^{3\rho}}\sum_{m=0}^{m_i}\dfrac{1}{\epsilon_{m_i}^{3\rho + 1}} &\overset{(a)}{\leq} 
\dfrac{K\log T}{(\frac{T^2}{K^2})^{\frac{3}{2}}}\bigg[ 1 + 2^{3\frac{1}{2} + 1}\dfrac{2^{ \frac{1}{2}\log_{2} \frac{T}{e}}-1}{2-1} \bigg] \\
%%%%%%%%%%%%%%%%%%%%%%%%%%%%%%%%
&\leq \dfrac{6 K^4\sqrt{T}\log T}{T^3} \leq \dfrac{6 K^4\log T}{T^{\frac{5}{2}}}
\end{align*}

where, in $(a)$ we substitute the values of $\rho$ and $\psi$. Substituting the values in Equation \ref{eq:contra:1} we get,
\begin{align}
& \dfrac{6K^{\frac{5}{2}}}{T^{\frac{9}{8}}} < \dfrac{6K^4\log T}{T^{\frac{5}{2}}}\nonumber\\
& \dfrac{T^{\frac{11}{8}}}{\log T} < K^{\frac{3}{2}}\nonumber\\
& \dfrac{T^{\frac{11}{8}}}{\sqrt{T}} \overset{(a)}{<} K^{1.5}\nonumber\\
& K^{2.1} \overset{(b)}{<} K^{1.5} \label{eq:ineq1}
\end{align}

where, $(a)$ occurs because of the inequality $\log T < \sqrt{T}$ and $(b)$ happens because of the condition that $T\geq K^{2.4}$. But the inequality \ref{eq:ineq1} is not possible for any $K\geq 2$. So clearly we can conclude that,
\begin{align*}
\dfrac{6K}{(\psi T \epsilon_{m_i})^{\frac{3\rho}{2}}} > \dfrac{32K\log T}{(\psi T)^{3\rho}}\sum_{m=0}^{m_i}\dfrac{1}{\epsilon_{m_i}^{3\rho + 1}}
\end{align*}

\hfill $\blacksquare$	
\end{proof}

\subsection{Proof of Lemma 7}
\label{App:Lemma:7}
\begin{lemma}
For all bounded rewards in $[0,1]$, $\dfrac{\Delta_i}{4} \geq \dfrac{\Delta_i}{16\sigma_i^2 + 1} $.
\end{lemma}

\begin{proof}
Since all rewards are bounded in $[0,1]$, we know that $\sigma_i^2 \leq \frac{1}{4}$. Hence we can show that,
\begin{align*}
\dfrac{\Delta_i}{16\sigma_i^2 + 1} &\leq \dfrac{\Delta_i}{16.\frac{1}{4} + 1}\\
& \leq \dfrac{\Delta_i}{5}
\end{align*}
 
But for all $\Delta_i \in [0,1]$ we know that $\dfrac{\Delta_{i}}{5} \leq \dfrac{\Delta_{i}}{4} $. Hence, 
$\dfrac{\Delta_i}{4} \geq \dfrac{\Delta_i}{16\sigma_i^2 + 1} $.

\hfill $\blacksquare$	
\end{proof}



\subsection{Proof of Lemma 8}
\label{App:Lemma:8}
\begin{lemma}
For two integer constants $c_1$ and $c_2$, if $c_1 < 20 c_2$ then,
\begin{align*}
c_1 \dfrac{16\sigma_i^2 + 1}{\Delta_i}\log\bigg( \dfrac{T\Delta_i^2}{K}\bigg) < c_2 \dfrac{\sigma_i^2}{\Delta_i}\log\bigg( \dfrac{T\Delta_i^2}{K}\bigg)
\end{align*}
 
\end{lemma}

\begin{proof}
We again prove this by contradiction. Suppose, 
\begin{align*}
c_1 \dfrac{16\sigma_i^2 + 1}{\Delta_i}\log\bigg( \dfrac{T\Delta_i^2}{K}\bigg) > c_2 \dfrac{\sigma_i^2}{\Delta_i}\log\bigg( \dfrac{T\Delta_i^2}{K}\bigg)
\end{align*}

Further reducing the above two terms we can show that, 

\begin{align*}
& 16c_1\sigma_i^2 + c_1 > c_2\sigma_i^2\\
& 16c_1.\dfrac{1}{4} + c_1 > \dfrac{c_2}{4}\\
& 20 c_1 > c_2
\end{align*}

But, we already know that $20 c_1 < c_2$. Hence, 
\begin{align*}
c_1 \dfrac{16\sigma_i^2 + 1}{\Delta_i}\log\bigg( \dfrac{T\Delta_i^2}{K}\bigg) < c_2 \dfrac{\sigma_i^2}{\Delta_i}\log\bigg( \dfrac{T\Delta_i^2}{K}\bigg)
\end{align*}

\hfill $\blacksquare$	
\end{proof}


\subsection{Proof of Lemma 9}
\label{App:Lemma:9}
\begin{lemma}
If $m_*$ be the round that the optimal arm $*$ gets eliminated, then we can show that the regret is upper bounded by,

\begin{align*}
\sum_{m_{*}=0}^{max_{j\in \A^{'}}m_{j}}\sum_{i\in \A^{''}:m_{i}>m_{*}}\bigg(\dfrac{388 K}{(\psi  T\epsilon_{m_{*}})^{\frac{3\rho}{2}}} \bigg).T\max_{j\in \A^{''}:m_{j}\geq m_{*}}{\Delta}_{j} \\
%%%%%%%%%%%%%%%%%%%%%%%%
 \leq\sum_{i\in \A^{'}}\dfrac{C_2^{'} K^{\frac{5}{2}}}{\sqrt{T\Delta_i}} +\sum_{i\in \A^{''}\setminus \A^{'}}\dfrac{C_2^{'} K^{\frac{5}{2}}}{\sqrt{T b}}
\end{align*}

\end{lemma}

\begin{proof}
\begin{align*}
&\sum_{m_{*}=0}^{max_{j\in \A^{'}}m_{j}}\sum_{i\in \A^{''}:m_{i}>m_{*}}\bigg(\dfrac{388 K}{(\psi  T\epsilon_{m_{*}})^{\frac{3\rho}{2}}} \bigg).T\max_{j\in \A^{''}:m_{j}\geq m_{*}}{\Delta}_{j}\\
%%%%%%%%%%%%%%%%%%%%%%%%%%%%
&\leq\sum_{m_{*}=0}^{max_{j\in \A^{'}}m_{j}}\sum_{i\in \A^{''}:m_{i}>m_{*}}\bigg(\dfrac{388 K\sqrt{4}}{(\psi  T\epsilon_{m_{*}})^{\frac{3\rho}{2}}} \bigg).T.4\sqrt{\epsilon_{m_{*}}}\\
%%%%%%%%%%%%%%%%%%%%%%%%%%%%
&\leq\sum_{m_{*}=0}^{max_{j\in \A^{'}}m_{j}}\sum_{i\in \A^{''}:m_{i}>m_{*}}C_2 K\bigg(\dfrac{T^{1-\frac{3\rho}{2}}}{\psi^{\frac{3\rho}{2}}\epsilon_{m_{*}}^{\frac{3\rho}{2}-\frac{1}{2}}} \bigg)\\
%%%%%%%%%%%%%%%%%%%%%%%%%%%%
&\leq\sum_{i\in \A^{''}:m_{i}>m_{*}}\sum_{m_{*}=0}^{\min{\lbrace m_{i},m_{b}\rbrace}}\bigg(\dfrac{C_2 K T^{1-\frac{3\rho}{2}}}{\psi^{\frac{3\rho}{2}}2^{-(\frac{3\rho}{2} -\frac{1}{2})m_{*}}} \bigg)\\
%%%%%%%%%%%%%%%%%%%%%%%%%%%%
&\leq\sum_{i\in \A^{'}}\bigg(\dfrac{C_2 K T^{1-\frac{3\rho}{2}}}{\psi^{\frac{3\rho}{2}}2^{-(\frac{3\rho}{2} -\frac{1}{2})m_{*}}} \bigg)+\sum_{i\in \A^{''}\setminus \A^{'}}\bigg(\dfrac{C_2 K T^{1-\frac{3\rho}{2} }}{\psi^{\frac{3\rho}{2}}2^{-(\frac{3\rho}{2} -\frac{1}{2})m_{b}}} \bigg)\\
%%%%%%%%%%%%%%%%%%%%%%%%%%%%
&\leq\sum_{i\in \A^{'}}\bigg(\dfrac{C_2 K T^{1-\frac{3\rho}{2}}.2^{\frac{\frac{3\rho}{2}}{2}-\frac{1}{4}}}{\psi^{\frac{3\rho}{2}}\Delta_{i}^{\frac{3\rho}{2} -\frac{1}{2}}} \bigg)+\sum_{i\in \A^{''}\setminus \A^{'}}\bigg(\dfrac{C_2 K T^{1-\frac{3\rho}{2}}.2^{\frac{\frac{3\rho}{2}}{2}-\frac{1}{4}}}{\psi^{\frac{3\rho}{2}}b^{\frac{3\rho}{2} -\frac{1}{2}}} \bigg)\\
%%%%%%%%%%%%%%%%%%%%%%%%%%%%
&\leq\sum_{i\in \A^{'}}\bigg(\dfrac{ C_2 K 2^{\frac{\frac{3\rho}{2}}{2}+\frac{19}{4}}.T^{1-\frac{3\rho}{2} } }{\psi^{\rho}\Delta_{i}^{2\frac{3\rho}{2} -1}} \bigg)+\sum_{i\in \A^{''}\setminus \A^{'}}\bigg(\dfrac{C_2 K 2^{\frac{\frac{3\rho}{2}}{2}+\frac{19}{4}}.T^{1-\frac{3\rho}{2}} }{\psi^{\frac{3\rho}{2} }b^{2\frac{3\rho}{2}-1}} \bigg)\\
%%%%%%%%%%%%%%%%%%%%%%%%%%%%
&\overset{(a)}{\leq}\sum_{i\in \A^{'}}\bigg(\dfrac{C_2^{'} K .T^{1-\frac{3}{4}}}{(\frac{T}{K^2})^{\frac{3}{4}}\Delta_{i}^{2.\frac{3}{4} -1}} \bigg)+\sum_{i\in \A^{''}\setminus \A^{'}}\bigg(\dfrac{C_2^{'} K T^{1-\frac{3}{4}}}{(\frac{T}{K^2})^{\frac{3}{4}}b^{2.\frac{3}{4}-1}} \bigg)\\
%%%%%%%%%%%%%%%%%%%%%%%%%%%%
&\leq\sum_{i\in \A^{'}}\dfrac{C_2^{'} K^{\frac{5}{2}}}{\sqrt{T\Delta_i}} +\sum_{i\in \A^{''}\setminus \A^{'}}\dfrac{C_2^{'} K^{\frac{5}{2}}}{\sqrt{T b}}
%%%%%%%%%%%%%%%%%%%%%%%%%%%%
%& = \sum_{i\in \A^{'}}\bigg(\dfrac{ C_{2}(\rho) T^{1-\rho}}{\Delta_{i}^{2\rho-1}} \bigg)+\sum_{i\in \A^{''}\setminus \A^{'}}\bigg(\dfrac{C_{2}(\rho)T^{1-\rho}}{b^{2\rho -1}} \bigg) \text{, where } C_2(x) = \frac{2^{\frac{x}{2}+\frac{19}{4}}}{\psi^{x}}
\end{align*}
In the above simplification, $(a)$ is obtained by substituting the values of $\psi$ and $\rho$.

\hfill $\blacksquare$	
\end{proof}

\subsection{Proof of Corollary 1}
\label{App:Corollary:1}
\begin{proof}
\label{Proof:Corollary:1}
From \cite{bubeck2011pure}  we know that the function $x\in [0,1]\mapsto x\exp(-Cx^2)$ is  decreasing on $\left[\frac{1}{\sqrt{2C}},1\right ]$ for any $C>0$. Thus, we take $C=\left\lfloor \frac{T}{e}\right\rfloor$ and choose  $\Delta_{i}=\Delta=\sqrt{\frac{K\log K}{T}}>\sqrt{\frac{e}{T}}$ for all $i$.

First, let us recall the result in Theorem \ref{Result:Theorem:1} below:
\begin{align*}
\E [R_{T}] \leq &\sum\limits_{i\in \A :\Delta_{i} > b}\bigg\lbrace \dfrac{C_0 K^{\frac{5}{2}}}{\sqrt{T\Delta_i}} + \bigg(\Delta_{i}+\dfrac{8\sigma_i^2\log{(\frac{T\Delta_{i}^{2}}{K})}}{\Delta_{i}}\bigg)\bigg \rbrace\\ 
  & +\sum\limits_{i\in \A :0 < \Delta_{i}\leq b} \dfrac{C_2^{'} K^{\frac{5}{2}}}{\sqrt{T\Delta_i}} + \max_{i\in \A :0 < \Delta_{i}\leq b}\Delta_{i}T.
\end{align*}

Now,  with  $\Delta_i =\Delta = \sqrt{\frac{K\log K}{T}}>\sqrt{\frac{e}{T}}$ we obtain,
	\begin{align*}
	&\sum_{i\in \A :\Delta_{i} > b}\dfrac{8\sigma_i^2\log{(\frac{T\Delta_{i}^{2}}{K})}}{\Delta_{i}} \leq  \dfrac{8\sigma_i^2 K\sqrt{T}\log{(T\dfrac{K(\log K)}{T K})}}{\sqrt{K\log K}}\\ 
	&\leq  \dfrac{8\sigma_i^2\sqrt{KT}\log{(\log K)}}{\sqrt{\log K}}
	\overset{(a)}{\leq} 2\sigma_i^2\sqrt{KT} 
	\end{align*}		
	where $(a)$ follows from the identity $\dfrac{\log{(\log K)}}{\sqrt{\log K}}\leq 1$ for $K\geq 2$. 
	
For the term $\sum\limits_{i\in \A :\Delta_{i} > b}\dfrac{C_0 K^{\frac{5}{2}}}{\sqrt{T\Delta_i}}$ by substituting the value of $\Delta_i=\Delta=\sqrt{\dfrac{K\log K}{T}}$ we get,
\begin{align*}
\dfrac{C_0 K^{\frac{5}{2}+1}}{\sqrt{T\Delta_i}} &\leq \dfrac{C_0 K^{\frac{5}{2}}}{\sqrt{T\sqrt{\dfrac{K\log K}{T}}}} \\
%%%%%%%%%%%%%%%%%%%%%%%%%%
\leq \dfrac{C_0 K^{\frac{5}{2}+1}}{(KT\log K)^{\frac{1}{4}}} \leq \dfrac{C_0 K^3}{T^{\frac{1}{4}}}
\end{align*}	
	
Similarly for the term $\sum\limits_{i\in \A :0 < \Delta_{i}\leq b} \dfrac{C_2^{'} K^{\frac{5}{2}}}{\sqrt{T\Delta_i}}$ we can show that,
\begin{align*}
\sum\limits_{i\in \A :0 < \Delta_{i}\leq b} \dfrac{C_2^{'} K^{\frac{5}{2}}}{\sqrt{T\Delta_i}} &\leq \dfrac{C_2^{'} K^{\frac{5}{2}+1}}{\sqrt{T\sqrt{\dfrac{e}{T}}}} \leq \dfrac{C_2^{'} K^3}{T^{\frac{1}{4}}} 
\end{align*}

Thus, the total worst case gap-independent bound is given by
	\begin{align*}
	\E[R_{T}]\leq  \dfrac{C_3 K^3}{T^{\frac{1}{4}}} + 8\sigma_i^2\sqrt{KT}.
	\end{align*}	
	
where $C_3$ is an integer constant such that $C_3= C_0 + C_2^{'} $

\hfill $\blacksquare$	
\end{proof}