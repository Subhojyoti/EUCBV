%\subsection{Lemma 1}
%\label{sec:proofTheorem:Lemma1}
We first present the following technical lemma that is required  to prove the result in Theorem \ref{Result:Theorem:1}.

\begin{lemma}
\label{proofTheorem:Lemma:1}
If $T\geq K^{2.4}$, $\psi=\dfrac{T}{ K^2}$, $\rho=\dfrac{1}{2}$ and $m\leq \dfrac{1}{2} \log_2(\dfrac{T}{e}) $, then,
\begin{align*}
\dfrac{\rho m \log(2)}{\log(\psi T) - 2m\log( 2)} \leq \frac{3}{2}
\end{align*}
\end{lemma}
%\begin{proof}
%The proof is given in Section \ref{sec:proofTheorem:Lemma1}.
%\end{proof}
\begin{proof}
We are going to prove this by contradiction. Let's say,
\begin{align*}
 & \dfrac{\rho m \log(2)}{\log(\psi T) - 2m\log( 2)} \geq \frac{3}{2} \\
 &\Rightarrow 2\rho m \log(2) \geq 3\log(\psi T) - 6m\log( 2) \\
 &\Rightarrow 2\rho m \log(2) \overset{(a)}{\geq} 6\log(\frac{T}{K}) - 6m\log( 2) \\
 &\Rightarrow 7 m \log(2) + 6\log(K) \overset{(b)}{\geq} 6\log(T) \\
 &\Rightarrow 3.5 \log(2) \log_2(\frac{T}{e}) + 6\log(K) \overset{(c)}{\geq} 6\log(T) \\
 &\Rightarrow \dfrac{3.5 \log (2) \log (\frac{T}{e})}{\log(2)} + 6\log(K) \geq 6\log(T)\\
 &\Rightarrow 3.5 \log(T) + 6\log K - 3.5 \geq 6\log (T)\\
 &\Rightarrow 6\log K \geq 2.5 \log T + 3.5 
\end{align*}
In the above inequalities, $(a)$ happens because $\psi=\frac{T}{K^2}$, $(b)$ occurs as $\rho=\frac{1}{2}$ and $(c)$ happens because $m\leq \frac{1}{2} \log_2(\frac{T}{e})$. But, for $T\geq K^{2.4}$, we can see that $6\log K \geq 2.5 \log T + 3.5$ is clearly not possible. Hence, $\dfrac{\rho m \log(2)}{\log(\psi T) - 2m\log( 2)} \leq \dfrac{3}{2}$.
\end{proof}

\subsection*{Proof of Theorem 1}
\label{sec:proofTheorem:Theorem1}
\begin{proof}
Let, for each sub-optimal arm ${i}$, $m_{i}=\min{\lbrace m|\sqrt{4\epsilon_{m_i}} < \dfrac{\Delta_{i}}{4} \rbrace}$. Let $\A^{'}=\lbrace i\in \A: \Delta_{i} > b \rbrace$ and $\A^{''}=\lbrace i\in \A: \Delta_{i} > 0 \rbrace$. Also $z_{i}$ denotes total number of times an arm $i$ has been pulled. In the $m$-th round, $n_{m}$ denotes the number of pulls allocated to the surviving arms in $B_{m}$. \\
\subsection*{Case $a$: \textit{Some sub-optimal arm ${i}$ is not eliminated in round $m_{i}$ or before and the optimal arm ${*}\in B_{m_{i}}$}}

An arbitrary sub-optimal arm ${i}\in \A^{'}$ can get eliminated only when the event,
	\begin{align}
	\hat{r}_{i} \leq r_{i} + c_{i} \text{ and } \label{eq:armelim-casea}
 	\hat{r}^{*} \geq r^{*} - c^{*}
	\end{align}
takes place. So to bound the regret we need to bound the probability of the complementary event of these two conditions. We denote $c_{i} = \sqrt{\frac{\rho (\hat{v}_i + 2) \log (\psi T\epsilon_{m_{i}})}{4 n_{m_i}}}$ for the $i$-th arm. Note that as arm elimination condition is being checked in every timestep, in the $m_i$-th round whenever $z_i \geq n_{m_{i}}=\dfrac{\log{(\psi T\epsilon_{m_{i}}^{2})}}{2\epsilon_{m_{i}}}$ we have, 
%any arm $i$ cannot be pulled more than $z_i= n_{m_i}$ times or it will get eliminated. This is because in the $m_i$-th round $n_{m_{i}}=\dfrac{\log{(\psi T\epsilon_{m_{i}}^{2})}}{2\epsilon_{m_{i}}}$ and putting this in $c_{m_i}$ we get,
	\begin{align*}
	c_{i} &\leq \sqrt{\dfrac{\rho (\hat{v}_i + 2)\epsilon_{m_{i}}\log (\psi T\epsilon_{m_{i}})}{2\log(\psi T\epsilon_{m_{i}}^{2})}} \overset{(a)}{\leq} \sqrt{\dfrac{2.\rho\epsilon_{m_{i}}\log (\frac{\psi T\epsilon_{m_{i}}^{2}}{\epsilon_{m_{i}}})}{\log(\psi T\epsilon_{m_{i}}^{2})}} \\
	%%%%%%%%%%%%%%%%%%%%%%%%%%
	& = \sqrt{\dfrac{2\rho\epsilon_{m_{i}}\log (\psi T\epsilon_{m_{i}}^{2}) - 2\rho\epsilon_{m_{i}}\log (\epsilon_{m_{i}})}{\log(\psi T\epsilon_{m_{i}}^{2})}} 
	\leq  \sqrt{2\rho\epsilon_{m_{i}} - \dfrac{2\rho\epsilon_{m_i}\log(\frac{1}{2^{m_i}})}{\log(\psi T \frac{1}{2^{2m_i}})}} \\
	%%%%%%%%%%%%%%%%%%%%%%%%%%
	&\leq \sqrt{2\rho\epsilon_{m_{i}} + \dfrac{2\rho\epsilon_{m_i}\log(2^{m_i})}{\log(\psi T) - \log( 2^{2m_i})}}  \leq \sqrt{2\rho\epsilon_{m_{i}} + \dfrac{2\rho\epsilon_{m_i}m_i \log(2)}{\log(\psi T) - 2m_i\log( 2)}} \\ 
	%%%%%%%%%%%%%%%%%%%%%%%%%%
	 & \overset{(b)}{\leq} \sqrt{2\rho\epsilon_{m_{i}} + 2.\frac{3}{2}\epsilon_{m_i}} 
	  < \sqrt{4\epsilon_{m_i}} 
	  < \dfrac{\Delta_{i}}{4} 
	\end{align*}
In the above $(a)$ happens because $\hat{v}_i \in [0,1]$, $\rho =\frac{1}{2}$ and $(b)$ occurs by applying the result from Lemma \ref{proofTheorem:Lemma:1}. Similarly, we can also show that in the $m_i$-th round $c^* < \frac{\Delta_i}{4}$.
Again, in the $m_i$-th round a sub-optimal arm ${i} \in \A^{'}$ gets eliminated as, 
  \begin{align*}
\hat{r}_{i} + c_{i}&\leq r_{i} + 2c_{i} 
= r_{i} + 4c_{i} - 2c_{i} \\
 &< r_{i} + \Delta_{i} - 2c_{i}
 \leq r^{*} -2c^{*} 
 \leq \hat{r}^{*} - c^{*}
  \end{align*}
	\noindent
	Thus, the probability that a bad arm is not eliminated correctly in the $m_i$-th round (or before) is given by ,
\begin{align}
\mathbb{P}(\hat{r}_{i}> r_{i} + c_{i})
&\leq \mathbb{P}\left( \hat{r}_{i} > r_{i}+ \bar{c}_i\right) 
+ \mathbb{P}\left( \hat{v}_{i}\geq \sigma_{i}^{2}+\sqrt{\epsilon_{m_{i}}}\right)\label{eq:prob_eq2}
\end{align}
where 
\begin{align*}
\bar{c}_i=\sqrt{\dfrac{\rho (\sigma_{i}^{2}+\sqrt{\epsilon_{m_{i}}} + 2)\log(\psi T\epsilon_{m_{i}})}{4n_{m_i}}}
\end{align*}
%\end{small}
Note that, substituting $ n_{m_i} \geq \frac{\log{(\psi T\epsilon_{m_{i}})}}{2\epsilon_{m_{i}}}$, $\bar{c}_i$ can be simplified to obtain,
\begin{align}
\bar{c}_i
\leq \sqrt{\dfrac{\rho\epsilon_{m_{i}}(\sigma_{i}^{2}+\sqrt{\epsilon_{m_{i}}} + 2)}{2}}\leq \sqrt{ \epsilon_{m_{i}}}.
\label{si_bar_equn}
\end{align}

The first term in the LHS of (\ref{eq:prob_eq2}) can be bounded using the Chernoff-Hoeffding bound as below:
\begin{align}
\mathbb{P}\left( \hat{r}_{i} > r_{i}+ \bar{c}_i\right)\nonumber 
\le \exp\left(- (\bar{c}_i)^2 z_{i} \right)\nonumber 
%%%%%%%%%%%%%%%
 &\le \exp\left(- \rho (\sigma_{i}^{2}+\sqrt{\epsilon_{m_{i}}} + 2)\log(\psi  T\epsilon_{m_{i}})\right)\nonumber \\
%%%%%%%%%%%%%%%
& \overset{(a)}{\leq} \exp\left(- \rho \log(\psi  T\epsilon_{m_{i}})\right) 
%%%%%%%%%%%%%%%
\le \dfrac{1}{(\psi  T\epsilon_{m_{i}})^{\rho}}
\label{lhs1_equn}
\end{align}
where, $(a)$ occurs because $(\sigma_{i}^{2}+\sqrt{\rho\epsilon_{m_{i}}} + 2) \geq 1$.

 
The second term in the LHS of (\ref{eq:prob_eq2}) can be simplified as follows:
\begin{align}
&\mathbb{P}\bigg\lbrace \hat{v}_{i}\geq \sigma_{i}^{2}+\sqrt{\epsilon_{m_{i}}}\bigg\rbrace\nonumber
%%%%%%%%%%%%%%%%%%
\leq \mathbb{P}\bigg\lbrace \dfrac{1}{n_{i}}\sum_{t=1}^{n_{i}}(X_{i,t}-r_{i})^{2}-(\hat{r}_{i}-r_{i})^{2}\geq \sigma_{i}^{2}+\sqrt{\epsilon_{m_{i}}}\bigg\rbrace\nonumber\\
%%%%%%%%%%%%%%%%%%
&\leq \mathbb{P}\bigg\lbrace \dfrac{\sum_{t=1}^{n_{i}}(X_{i,t}-r_{i})^{2}}{n_{i}}\geq \sigma_{i}^{2}+\sqrt{\epsilon_{m_{i}}} \bigg\rbrace\nonumber
%%%%%%%%%%%%%%%%%%
\overset{(a)}{\leq} \mathbb{P}\bigg\lbrace \dfrac{\sum_{t=1}^{n_{i}}(X_{i,t}-r_{i})^{2}}{n_{i}}\geq \sigma_{i}^{2} + \bar{c}_i\bigg\rbrace \nonumber\\
%%%%%%%%%%%%%%%%%%
&\overset{(b)}{\leq} \exp\left(- \rho (\sigma_{i}^{2}+\sqrt{\epsilon_{m_{i}}} + 2)\log(\psi  T\epsilon_{m_{i}})\right)
%%%%%%%%%%%%%%%%%
\le \dfrac{1}{(\psi  T\epsilon_{m_{i}})^{\rho}}
\label{lhs2_equn}
\end{align}
where inequality $(a)$ is obtained using (\ref{si_bar_equn}), while $(b)$ follows from the Chernoff-Hoeffding bound.
  
Thus, using (\ref{lhs1_equn}) and (\ref{lhs2_equn}) in (\ref{eq:prob_eq2}) we obtain $\mathbb{P}(\hat{r}_{i}> r_{i} + c_{i})\le \dfrac{2}{(\psi  T\epsilon_{m_{i}})^{\rho}}$. Similarly, $\mathbb{P}\lbrace\hat{r}^{*}\leq r^{*} - c^{*}\rbrace \leq \dfrac{2}{(\psi  T\epsilon_{m_{i}})^{\rho}}$. Summing the two up, the probability that a sub-optimal arm ${i}$ is not eliminated on or before $m_{i}$-th round is  $\bigg(\dfrac{4}{(\psi T\epsilon_{m_{i}})^{\rho}} \bigg)$.

Summing up over all arms in $\A^{'}$ and bounding the regret for each arm $i\in \A^{'}$ trivially by $T\Delta_{i}$, we obtain
   \begin{align*}
&\sum_{i\in \A^{'}}\bigg(\dfrac{4T\Delta_{i}}{(\psi T\epsilon_{m_{i}})^{\rho}}\bigg)
\leq\sum_{i\in \A^{'}}\bigg(\dfrac{4T\Delta_{i}}{(\psi T\dfrac{\Delta_{i}^{2}}{4.16})^{\rho}}\bigg)
\leq \sum_{i\in \A^{'}}\bigg(\dfrac{2^{2+2\rho}.16^{\rho}T^{1-\rho}}{\psi^{\rho}\Delta_{i}^{2\rho -1}}\bigg)\\  
& \overset{(a)}{\leq} \sum_{i\in \A^{'}}\bigg(\dfrac{2^{2+1}.16^{\frac{1}{2}}T^{1-\frac{1}{2}}}{(\frac{T}{K^2})^{\frac{1}{2}}\Delta_{i}^{2.\frac{1}{2} -1}}\bigg) = \sum_{i\in \A^{'}}  32 K  
%& =\sum_{i\in \A^{'}}\bigg(\dfrac{C_{1}(\rho)T^{1-\rho}}{\Delta_{i}^{2\rho -1}}\bigg) \text{, where } C_1(x) = \frac{2^{2+6x}}{\psi^{x}}
   \end{align*}
   Here in $(a)$ we substitute the values of $\rho$ and $\psi$.

\subsection*{Case $b$: \textit{An arm ${i}\in B_{m_i}$ is eliminated in round $m_{i}$ or before or there is no $*\in B_{m_i}$}}

\subsubsection*{Case $b1$: \textit{${*}\in B_{m_{i}}$ and each ${i}\in \A^{'}$ is  eliminated on or before $m_{i}$ } }

	\paragraph*{} Since we are eliminating a sub-optimal arm ${i}$ on or before round $m_{i}$, it is pulled no longer than, 
 \begin{align*}
 z_{i} < \bigg\lceil\dfrac{\log{(\psi T\epsilon_{m_{i}}^{2})}}{2\epsilon_{m_{i}}}\bigg\rceil
 \end{align*}
%\hspace*{4em}
%%$, since $\sqrt{\rho_{a}\epsilon_{m_{i}}}\leq\dfrac{\Delta_{i}}{2}
So, the total contribution of ${i}$  till round $m_{i}$ is given by, 
\begin{align*}
&\Delta_{i}\bigg\lceil\dfrac{\log{(\psi T\epsilon_{m_{i}}^{2})}}{2\epsilon_{m_{i}}}\bigg\rceil
\leq\Delta_{i}\bigg\lceil\dfrac{\log{(\psi T(\dfrac{\Delta_{i}}{16 \times 256})^{4})}}{2(\dfrac{\Delta_{i}}{4\sqrt{4}})^{2}}\bigg\rceil \text{, since } \sqrt{4\epsilon_{m_{i}}} < \dfrac{\Delta_{i}}{4}\\
&\leq\Delta_{i}\bigg(1+\dfrac{32\log{(\psi T(\dfrac{\Delta_{i}^{4}}{16384})}}{\Delta_{i}^{2}}\bigg)
\leq \Delta_{i}\bigg(1+\dfrac{32\log{(\psi T\Delta_{i}^{4})}}{\Delta_{i}^{2}}\bigg) 
\end{align*} 

Summing over all arms in $\A^{'}$ the total regret is given by, 
\begin{align*}
\sum_{i\in \A^{'}}\Delta_{i}\bigg(1+\dfrac{32\log{(\psi T\Delta_{i}^{4}})}{\Delta_{i}^{2}}\bigg) = \sum_{i\in \A^{'}}\bigg(\Delta_{i} +\dfrac{32\log{(\psi T\Delta_{i}^{4}})}{\Delta_{i}}\bigg) \\
\overset{(a)}{\leq} \sum_{i\in \A^{'}} \Delta_{i}\bigg(1+\dfrac{64\log{( \frac{T\Delta_{i}^{2}}{K})}}{\Delta_{i}^{2}}\bigg)
\end{align*}

We obtain $(a)$ by substituting the value of $\psi$. 

\subsubsection*{Case $b2$: \textit{Optimal arm ${*}$ is eliminated by a sub-optimal arm  }}


	Firstly, if conditions of Case $a$ holds then the optimal arm ${*}$ will not be eliminated in round $m=m_{*}$ or it will lead to the contradiction that $r_{i}>r^{*}$. In any round $m_{*}$, if the optimal arm ${*}$ gets eliminated then for any round from $1$ to $m_{j}$ all arms ${j}$ such that $m_{j}< m_{*}$ were eliminated according to assumption in Case $a$. Let the arms surviving till $m_{*}$ round be denoted by $\A^{'}$. This leaves any arm $a_{b}$ such that $m_{b}\geq m_{*}$ to still survive and eliminate arm ${*}$ in round $m_{*}$. Let such arms that survive ${*}$ belong to $\A^{''}$. Also maximal regret per step after eliminating ${*}$ is the maximal $\Delta_{j}$ among the remaining arms ${j}$ with $m_{j}\geq m_{*}$.  Let $m_{b}=\min\lbrace m|\sqrt{4\epsilon_{m}}<\dfrac{\Delta_{b}}{4}\rbrace$. Hence, the maximal regret after eliminating the arm ${*}$ is upper bounded by, 
\begin{align*}
&\sum_{m_{*}=0}^{max_{j\in \A^{'}}m_{j}}\sum_{i\in \A^{''}:m_{i}>m_{*}}\bigg(\dfrac{4}{(\psi  T\epsilon_{m_{*}})^{\rho}} \bigg).T\max_{j\in \A^{''}:m_{j}\geq m_{*}}{\Delta}_{j}\\
%%%%%%%%%%%%%%%%%%%%%%%%%%%%
&\leq\sum_{m_{*}=0}^{max_{j\in \A^{'}}m_{j}}\sum_{i\in \A^{''}:m_{i}>m_{*}}\bigg(\dfrac{4\sqrt{4}}{(\psi  T\epsilon_{m_{*}})^{\rho}} \bigg).T.4\sqrt{\epsilon_{m_{*}}}\\
%%%%%%%%%%%%%%%%%%%%%%%%%%%%
&\leq\sum_{m_{*}=0}^{max_{j\in \A^{'}}m_{j}}\sum_{i\in \A^{''}:m_{i}>m_{*}}32\bigg(\dfrac{T^{1-\rho}}{\psi^{\rho}\epsilon_{m_{*}}^{\rho-\frac{1}{2}}} \bigg)\\
%%%%%%%%%%%%%%%%%%%%%%%%%%%%
&\leq\sum_{i\in \A^{''}:m_{i}>m_{*}}\sum_{m_{*}=0}^{\min{\lbrace m_{i},m_{b}\rbrace}}\bigg(\dfrac{32T^{1-\rho}}{\psi^{\rho}2^{-(\rho -\frac{1}{2})m_{*}}} \bigg)\\
%%%%%%%%%%%%%%%%%%%%%%%%%%%%
&\leq\sum_{i\in \A^{'}}\bigg(\dfrac{32T^{1-\rho}}{\psi^{\rho}2^{-(\rho -\frac{1}{2})m_{*}}} \bigg)+\sum_{i\in \A^{''}\setminus \A^{'}}\bigg(\dfrac{32T^{1-\rho }}{\psi^{\rho}2^{-(\rho -\frac{1}{2})m_{b}}} \bigg)\\
%%%%%%%%%%%%%%%%%%%%%%%%%%%%
&\leq\sum_{i\in \A^{'}}\bigg(\dfrac{32T^{1-\rho}*2^{\frac{\rho}{2}-\frac{1}{4}}}{\psi^{\rho}\Delta_{i}^{\rho -\frac{1}{2}}} \bigg)+\sum_{i\in \A^{''}\setminus \A^{'}}\bigg(\dfrac{32T^{1-\rho_{a}}}{\psi^{\rho}b^{\rho -\frac{1}{2}}} \bigg)\\
%%%%%%%%%%%%%%%%%%%%%%%%%%%%
&\leq\sum_{i\in \A^{'}}\bigg(\dfrac{2^{\frac{\rho}{2}+\frac{19}{4}}.T^{1-\rho } }{\psi^{\rho}\Delta_{i}^{2\rho -1}} \bigg)+\sum_{i\in \A^{''}\setminus \A^{'}}\bigg(\dfrac{2^{\frac{\rho}{2}+\frac{19}{4}}.T^{1-\rho} }{\psi^{\rho }b^{2\rho_{a}-1}} \bigg)\\
%%%%%%%%%%%%%%%%%%%%%%%%%%%%
&\overset{(a)}{\leq}\sum_{i\in \A^{'}}\bigg(\dfrac{2^{\frac{1}{4}+\frac{19}{4}}.T^{1-\frac{1}{2} } }{(\frac{T}{K^2})^{\frac{1}{2}}\Delta_{i}^{2.\frac{1}{2} -1}} \bigg)+\sum_{i\in \A^{''}\setminus \A^{'}}\bigg(\dfrac{2^{\frac{1}{4}+\frac{19}{4}}.T^{1-\frac{1}{2}} }{(\frac{T}{K^2})^{\frac{1}{2} }b^{2.\frac{1}{2}-1}} \bigg)\\
%%%%%%%%%%%%%%%%%%%%%%%%%%%%
&\leq\sum_{i\in \A^{'}}32 K +\sum_{i\in \A^{''}\setminus \A^{'}}32 K
%%%%%%%%%%%%%%%%%%%%%%%%%%%%
%& = \sum_{i\in \A^{'}}\bigg(\dfrac{ C_{2}(\rho) T^{1-\rho}}{\Delta_{i}^{2\rho-1}} \bigg)+\sum_{i\in \A^{''}\setminus \A^{'}}\bigg(\dfrac{C_{2}(\rho)T^{1-\rho}}{b^{2\rho -1}} \bigg) \text{, where } C_2(x) = \frac{2^{\frac{x}{2}+\frac{19}{4}}}{\psi^{x}}
\end{align*}

In the above section, $(a)$ is obtained by substituting the values of $\psi$ and $\rho$. Summing up \textbf{Case a} and \textbf{Case b}, the total regret is given by,
\begin{align*}
% \E [R_{T}] \leq &\sum\limits_{i\in \A :\Delta_{i} > b}\bigg\lbrace\bigg(\dfrac{C_{1}(\rho)T^{1-\rho}}{\Delta_{i}^{2\rho -1}}\bigg) + \bigg(\Delta_{i}+\dfrac{32\log{(\psi  T\Delta_{i}^{4})}}{\Delta_{i}}\bigg) + \bigg(\dfrac{ C_{2}(\rho) T^{1-\rho}}{\Delta_{i}^{2\rho-1}} \bigg)\bigg \rbrace\\ 
%  & +\sum\limits_{i\in \A :0 < \Delta_{i}\leq b}\bigg(\dfrac{C_{2}(\rho)T^{1-\rho}}{b^{2\rho -1}} \bigg) + \max_{i\in \A :0 < \Delta_{i}\leq b}\Delta_{i}T
\E [R_{T}] \leq &\sum\limits_{i\in \A :\Delta_{i} > b}\bigg\lbrace 64 K + \bigg(\Delta_{i}+\dfrac{64\log{(\frac{T\Delta_{i}^{2}}{K})}}{\Delta_{i}}\bigg)\bigg \rbrace\\ 
  & +\sum\limits_{i\in \A :0 < \Delta_{i}\leq b} 32 K + \max_{i\in \A :0 < \Delta_{i}\leq b}\Delta_{i}T  
\end{align*}
\end{proof}


%%%%%%%%%%%%%%%%%%%%%%%%%%%%%%%%%%%%%%%%%%%%%%%%%%%%%
%Removed Corollary 1
%%%%%%%%%%%%%%%%%%%%%%%%%%%%%%%%%%%%%%%%%%%%%%%%%%%%%


%\subsection{Proof of Corollary 1}
%\label{sec:proofTheorem:Corollary1}
%\begin{proof}
%\label{Proof:Corollary:1}
%For proving this corollary we first state the result of Theorem \ref{Result:Theorem:1} below, 
%	
%\begin{align*}
%\E [R_{T}] \leq &\sum\limits_{i\in \A :\Delta_{i} > b}\bigg\lbrace\bigg(\dfrac{C_{1}(\rho)T^{1-\rho}}{\Delta_{i}^{2\rho -1}}\bigg) + \bigg(\Delta_{i}+\dfrac{32\log{(\psi  T\Delta_{i}^{4})}}{\Delta_{i}}\bigg) + \bigg(\dfrac{ C_{2}(\rho) T^{1-\rho}}{\Delta_{i}^{2\rho-1}} \bigg)\bigg \rbrace\\ 
%  & +\sum\limits_{i\in \A :0 < \Delta_{i}\leq b}\bigg(\dfrac{C_{2}(\rho)T^{1-\rho}}{b^{2\rho -1}} \bigg) + \max_{i\in \A :0 < \Delta_{i}\leq b}\Delta_{i}T
%\end{align*} 
%
%%\begin{align*}
%%\E [R_{T}] \leq &\sum\limits_{i\in \A :\Delta_{i} > b}\bigg\lbrace\bigg(\dfrac{C_{1}(\rho)T^{1-\rho}}{\Delta_{i}^{2\rho -1}}\bigg) + \bigg(\Delta_{i}+\dfrac{41\log{(\psi  T\Delta_{i}^{4})}}{\Delta_{i}}\bigg) + \bigg(\dfrac{ C_{2}(\rho) T^{1-\rho}}{\Delta_{i}^{2\rho-1}} \bigg)\bigg \rbrace\\ 
%%  & +\sum\limits_{i\in \A :0 < \Delta_{i}\leq b}\bigg(\dfrac{C_{2}(\rho)T^{1-\rho}}{b^{2\rho -1}} \bigg) + \max_{i\in \A :0 < \Delta_{i}\leq b}\Delta_{i}T
%%\end{align*}
%
%After putting the parameter values $\psi=\frac{T}{K^2}$ and $\rho=\frac{1}{2}$ in the above Theorem \ref{Result:Theorem:1} result we get,
%	\begin{align*}
%	\sum_{i\in \A :\Delta_{i} > b}\bigg(\dfrac{T^{1-\rho}}{\psi^{\rho}\Delta_{i}^{2\rho -1}} \bigg) &= \sum_{i\in \A :\Delta_{i} > b}\bigg(\dfrac{T^{1-\frac{1}{2}}2^{2+6.\frac{1}{2}}}{(\frac{T}{( K)^2})^{\frac{1}{2}}\Delta_{i}^{2.\frac{1}{2}-1}} \bigg) = \sum_{i\in \A :\Delta_{i} > b} 32 K 
%	\end{align*}
%	
%	
%	For the term involving arm pulls,
%	\begin{align*}
%	\sum_{i\in \A :\Delta_{i} > b}\dfrac{32\log{(\psi T\Delta_{i}^{4})}}{\Delta_{i}}=\sum_{i\in \A :\Delta_{i} > b}\dfrac{64\log{(\dfrac{T\Delta_{i}^{2}}{K})}}{\Delta_{i}}
%	%\approx \sum_{i\in A:\Delta_{i} > b}\dfrac{16\log{(T\dfrac{\Delta_{i}^{2}}{\log K})}}{\Delta_{i}}
%	\end{align*}		
%
%	Lastly we can bound the error terms as, 
%	\begin{align*}
%	\sum\limits_{i\in \A :0 < \Delta_{i}\leq b}\bigg(\dfrac{T^{1-\rho}2^{\frac{\rho}{2}+\frac{19}{4}}}{\psi^{\rho }\Delta_{i}^{2\rho -1}} \bigg)= \sum\limits_{i\in \A :0 < \Delta_{i}\leq b} 32 K
%	\end{align*}
%
%	So, the total gap dependent regret bound for using both arm and cluster elimination comes of as
%	\begin{align*}
%	& \sum_{i\in \A :\Delta_{i} > b}\bigg\lbrace 96 K + \dfrac{64\log{(\dfrac{T\Delta_{i}^{2}}{ K})}}{\Delta_{i}}\bigg\rbrace 
% + \max\limits_{i\in \A :\Delta_{i}\leq b}\Delta_{i}T 
%	\end{align*}
%\end{proof}

%%%%%%%%%%%%%%%%%%%%%%%%%%%%%%%%%%%%%%%%%%%%%%%%%%%%%%%%%%%%%
%Corollary 2 moved to result
%%%%%%%%%%%%%%%%%%%%%%%%%%%%%%%%%%%%%%%%%%%%%%%%%%%%%%%%%%%%%


%\subsection{Proof of Corollary 2}
%\label{sec:proofTheorem:Corollary2}
%\begin{proof}
%\label{Proof:Corollary:2}
%In the non-stochastic scenario \cite{auer2002nonstochastic} showed that the bound on the cumulative regret can be $O\left(\sqrt{KT\log K}\right)$. But UCB1 suffered from a regret of order of  $O\left(\sqrt{KT\log T}\right)$  which is clearly improvable. Also we know from \citet{bubeck2011pure} that the function $x\in [0,1]\mapsto x\exp(-Cx^2)$ is  decreasing on $\left[\frac{1}{\sqrt{2C}},1\right ]$ for any $C>0$. So, we take $C=\left\lfloor \frac{T}{e}\right\rfloor$ and choose  $\Delta_{i}=\Delta=\sqrt{\frac{K\log K}{T}}>\sqrt{\frac{e}{T}}$ for all ${i:i\neq *}\in \A $.
%
%	We again state the result of Theorem \ref{Result:Theorem:1} below, 
%	
%\begin{align*}
%\E [R_{T}] \leq &\sum\limits_{i\in \A :\Delta_{i} > b}\bigg\lbrace\bigg(\dfrac{C_{1}(\rho)T^{1-\rho}}{\Delta_{i}^{2\rho -1}}\bigg) + \bigg(\Delta_{i}+\dfrac{32\log{(\psi  T\Delta_{i}^{4})}}{\Delta_{i}}\bigg) + \bigg(\dfrac{ C_{2}(\rho) T^{1-\rho}}{\Delta_{i}^{2\rho-1}} \bigg)\bigg \rbrace\\ 
%  & +\sum\limits_{i\in \A :0 < \Delta_{i}\leq b}\bigg(\dfrac{C_{2}(\rho)T^{1-\rho}}{b^{2\rho -1}} \bigg) + \max_{i\in \A :0 < \Delta_{i}\leq b}\Delta_{i}T
%\end{align*}
%%
%%	
%%	Taking into account Theorem \ref{Result:Theorem:1} for all $b\geq \sqrt{\dfrac{e}{T}}$,
%%
%%\begin{align*}
%%\E [R_{T}] \leq &\sum\limits_{i\in \A :\Delta_{i} > b}\bigg\lbrace\bigg(\dfrac{C_{1}(\rho)T^{1-\rho}}{\Delta_{i}^{2\rho -1}}\bigg) + \bigg(\Delta_{i}+\dfrac{41\log{(\psi  T\Delta_{i}^{4})}}{\Delta_{i}}\bigg) + \bigg(\dfrac{ C_{2}(\rho) T^{1-\rho}}{\Delta_{i}^{2\rho-1}} \bigg)\bigg \rbrace\\ 
%%  & +\sum\limits_{i\in \A :0 < \Delta_{i}\leq b}\bigg(\dfrac{C_{2}(\rho)T^{1-\rho}}{b^{2\rho -1}} \bigg) + \max_{i\in \A :0 < \Delta_{i}\leq b}\Delta_{i}T
%%\end{align*}
%
%Putting the parameter values $\psi=\frac{T}{K^2}$ and $\rho=\frac{1}{2}$ in the result of above Theorem \ref{Result:Theorem:1} we get,
%	\begin{align*}
%	\sum_{i\in \A :\Delta_{i} > b}\bigg(\dfrac{C_{1}(\rho)T^{1-\rho}}{\Delta_{i}^{2\rho -1}}\bigg) &= \sum_{i\in \A :\Delta_{i} > b}\bigg(\dfrac{T^{1-\frac{1}{2}}2^{2+6.\frac{1}{2}}}{(\frac{T}{( K)^2})^{\frac{1}{2}}\Delta_{i}^{2.\frac{1}{2}-1}} \bigg) \leq 32 K^2
%	\end{align*}
%	 
%	
%	For the term regarding number of pulls,
%	\begin{align*}
%	\sum_{i\in \A :\Delta_{i} > b}\dfrac{32\log{(\psi T\Delta_{i}^{4})}}{\Delta_{i}} &= \dfrac{32K\sqrt{T}\log{(T^{2}\dfrac{K^{2}(\log K)^{2}}{T^{2} K^2})}}{\sqrt{K\log K}} \leq  \dfrac{64\sqrt{KT}\log{(\log K)}}{\sqrt{\log K}}\\
%	&\leq 64\sqrt{KT} \text{ ,   as $\dfrac{\log{(\log K)}}{\sqrt{\log K}}\leq 1$}
%	\end{align*}		
%	%for $K\geq 3$
% 	Lastly we can bound the error terms as, 
%	\begin{align*}
%	\sum\limits_{i\in \A :0\leq\Delta_{i}\leq b}\bigg(\dfrac{T^{1-\rho }2^{\frac{\rho}{2} +\frac{19}{4}}}{\psi^{\rho }\Delta_{i}^{2\rho -1}} \bigg)=K\bigg(\dfrac{T^{1-\frac{1}{2}}2^{\frac{1}{4}+\frac{19}{4}}}{{(\frac{T}{K^2})^{\frac{1}{2}}}{(\Delta_{i})^{2*\frac{1}{2}-1}}} \bigg) < 32 K^2
%	\end{align*}	 	
% 	Similarly for the term,
% 	\begin{align*}
% 	\bigg(\dfrac{ C_{2}(\rho) T^{1-\rho}}{\Delta_{i}^{2\rho-1}} \bigg) \leq  \sum_{i\in \A : \Delta_{i} > b}\bigg(\dfrac{T^{1-\rho} 2^{\frac{\rho}{2}+\frac{19}{4}}}{(\psi^{\rho})\Delta_{i}^{2\rho -1}} \bigg) < 32  K^2
%	\end{align*} 	
% 	
%	So, the total bound for using both arm and cluster elimination cannot be worse than,
%	
%	\begin{align*}
%	\E[R_{T}]\leq 96 K^2 + 64\sqrt{KT}
%	\end{align*}		
%\end{proof}