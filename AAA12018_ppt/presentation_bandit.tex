%%%%%%%%%%%%%%%%%%%%%%%%%%%%%%%%%%%%%%%%%
% Beamer Presentation
% LaTeX Template
% Version 1.0 (10/11/12)
%
% This template has been downloaded from:
% http://www.LaTeXTemplates.com
%
% License:
% CC BY-NC-SA 3.0 (http://creativecommons.org/licenses/by-nc-sa/3.0/)
%
%%%%%%%%%%%%%%%%%%%%%%%%%%%%%%%%%%%%%%%%%

%----------------------------------------------------------------------------------------
%	PACKAGES AND THEMES
%----------------------------------------------------------------------------------------

\documentclass{beamer}


\mode<presentation> {

% The Beamer class comes with a number of default slide themes
% which change the colors and layouts of slides. Below this is a list
% of all the themes, uncomment each in turn to see what they look like.

%\usetheme{default}
%\usetheme{AnnArbor}
%\usetheme{Antibes}
%\usetheme{Bergen}
%\usetheme{Berkeley}
%\usetheme{Berlin}
%\usetheme{Boadilla}
%\usetheme{CambridgeUS}
%\usetheme{Copenhagen}
%\usetheme{Darmstadt}
%\usetheme{Dresden}
%\usetheme{Frankfurt}
%\usetheme{Goettingen}
%\usetheme{Hannover}
%\usetheme{Ilmenau}
%\usetheme{JuanLesPins}
%\usetheme{Luebeck}
\usetheme{Madrid}
%\usetheme{Malmoe}
%\usetheme{Marburg}
%\usetheme{Montpellier}
%\usetheme{PaloAlto}
%\usetheme{Pittsburgh}
%\usetheme{Rochester}
%\usetheme{Singapore}
%\usetheme{Szeged}
%\usetheme{Warsaw}

% As well as themes, the Beamer class has a number of color themes
% for any slide theme. Uncomment each of these in turn to see how it
% changes the colors of your current slide theme.

%\usecolortheme{albatross}
%\usecolortheme{beaver}
%\usecolortheme{beetle}
%\usecolortheme{crane}
%\usecolortheme{dolphin}
%\usecolortheme{dove}
%\usecolortheme{fly}
%\usecolortheme{lily}
%\usecolortheme{orchid}
%\usecolortheme{rose}
%\usecolortheme{seagull}
%\usecolortheme{seahorse}
%\usecolortheme{whale}
%\usecolortheme{wolverine}

%\setbeamertemplate{footline} % To remove the footer line in all slides uncomment this line
%\setbeamertemplate{footline}[page number] % To replace the footer line in all slides with a simple slide count uncomment this line

%\setbeamertemplate{navigation symbols}{} % To remove the navigation symbols from the bottom of all slides uncomment this line
}

\usepackage{macros}
%\usepackage{enumitem}
%\usepackage{biblatex}
%\addbibresource{ijcai17.bib}
%\usepackage{natbib}
%\usepackage[round]{natbib}
%\usepackage[comma,numbers,sort&compress]{natbib}

%----------------------------------------------------------------------------------------
%	TITLE PAGE
%----------------------------------------------------------------------------------------

\title[Efficient-UCBV: An Almost Optimal Algorithm using Variance Estimates]{Efficient-UCBV: An Almost Optimal Algorithm using Variance Estimates} % The short title appears at the bottom of every slide, the full title is only on the title page

\author{Subhojyoti Mukherjee (IIT Madras) \\Dr. K.P. Naveen (IIT Tirupati) \\ Dr. Nandan Sudarsanam (IIT Madras, RBC-DSAI)\\ Dr. Balaraman Ravindran (IIT Madras, RBC-DSAI)} % Your name
\institute[IIT Madras] % Your institution as it will appear on the bottom of every slide, may be shorthand to save space
{
IIT Madras \\ % Your institution for the title page
\medskip
%\textit{john@smith.com} % Your email address
}
%\date{\today} % Date, can be changed to a custom date
\date{Feb 6, 2018}

\begin{document}
\nocite{*}
\begin{frame}
\titlepage % Print the title page as the first slide
\end{frame}

\begin{frame}
\frametitle{Overview} % Table of contents slide, comment this block out to remove it
\tableofcontents % Throughout your presentation, if you choose to use \section{} and \subsection{} commands, these will automatically be printed on this slide as an overview of your presentation
\end{frame}

%----------------------------------------------------------------------------------------
%	PRESENTATION SLIDES
%----------------------------------------------------------------------------------------


%\section{Introduction}
%In this paper we deal with the stochastic multi-armed bandit (MAB) setting. Stochastic MABs are classic instances of sequential learning model where at each timestep a learner is exposed to a finite set of actions (or arms) and it has to choose one arm at a time. After choosing (or pulling) an arm the learner sees the reward for the arm as revealed by the environment. Each of these reward is an i.i.d random variable as sampled from the distribution associated with each arm. The mean of the reward distribution associated with an arm $i$ is denoted by $r_i$ whereas the mean of the reward distribution of the optimal arm $*$ is denoted by $r^*$ such that $r_i < r^*, \forall i\in \A$. One of the fundamental assumptions in stochastic MAB is that the distribution associated with each arm does not change over the entire time horizon $T$. The objective in the stochastic bandit problem is to minimize the cumulative regret, which is defined as follows:
\begin{align*}
R_{T}=r^{*}T - \sum_{i\in A} r_{i}n_{i}(T),
\end{align*}
where $T$ is the number of timesteps, $n_{i}(T)=\sum_{j=1}^T I(I_j=i)$ is the number of times the algorithm has chosen arm $i$ up to timestep $T$.
The expected regret of an algorithm after $T$ timesteps can be written as,
\begin{align*}
\E[R_{T}]= \sum_{i=1}^K \E[n_i (T)] \Delta_i,
\end{align*}
where $\Delta_{i}=r^{*}-r_{i}$ denotes the gap between the means of the optimal arm and the $i$-th arm.

\section{Stochastic Multi-Armed Bandit Problem}
\begin{frame}
\frametitle{Stochastic Multi-Armed Bandit Problem (SMAB)}
\begin{itemize}
%\item<1-> The thresholding bandit problem falls under the broad area of stochastic multi-armed bandit problem.
\item<1-> A finite set of actions or arms belonging to set $\A$ such that $|\A|=K$. 
\item<2-> The rewards for each of the arms, $X_{i,t}\sim^{i.i.d} D_i$. 
%\item<2-> The rewards for each of the arms are i.i.d random variables drawn from distribution specific to the arm which are fixed throughout the time horizon denoted by $T$.
\item<3-> The learner does not know the mean $r_{i}$  or the variance $\sigma_i^2$ of the distribution $D_i,\forall i\in \A$. 
%\item<5-> The distributions for each of the arms are fixed throughout the time horizon denoted by $T$. 
\item<4-> \textbf{Vital Assumption:} $D_i,\forall i\in\A$ are fixed throughout the time horizon denoted by $T$.
\end{itemize}
\end{frame}

%\begin{frame}
%\frametitle{Stochastic Multi-Armed Bandit Problem (SMAB)}
%\begin{itemize}
%\item<1-> The distributions for each of the arms are fixed throughout the time horizon denoted by $T$. 
%\item<2-> The estimated reward $\hat{r}_{i}=\frac{1}{n_{i}}\sum_{z=1}^{n_i} X_{i,z}$.
%\item<3-> The more we pull arm $i$ the closer $\hat{r}_i$ gets to $r_i$.
%\item<4-> Due to the uncertainty in $\hat{r}_i$ we have to carefully conduct exploration.
%\end{itemize}
%\end{frame}




\section{Problem Definition of SMAB}
\begin{frame}
\frametitle{Problem Definition of SMAB}
\begin{itemize}
\item<1-> \textbf{Primary aim:} Minimize the cumulative regret by identifying the arm whose expected mean is $r^*$ such that $r^* > r_i,\forall i\in\A$.
\item<2-> \textbf{Condition:} This has to be achieved within a finite $T$ timesteps.
\item<3-> The expected regret of an algorithm after $T$ timesteps is give by,
\begin{align*}
\E[R_{T}]= \sum_{i=1}^{K} \E[z_i (T)] \Delta_i,
\end{align*}
where $\Delta_{i}=r^{*}-r_{i}$ is the gap.
\end{itemize}
\end{frame}

%\begin{frame}
%\frametitle{Problem Definition of SMAB}
%\begin{itemize}
%\item<1-> We define the set $S_{\tau}=\lbrace i\in \mathcal{A}: r_{i}\geq \tau \rbrace$. 
%%Note that, $S_\tau$ is the set of all arms whose reward mean is greater than $\tau$. Let 
%\item<2-> $S_\tau^c$ denote the complement of $S_\tau$, i.e.,  $S_{\tau}^{c}=\lbrace i\in \mathcal{A}: r_{i} < \tau \rbrace$. 
%\item<3-> Let $\hat{S}_{\tau}$ denote the recommendation of a learning algorithm after $T$ time units of exploration, while $\hat{S}_{\tau}^c$ denotes its complement.
%
%%\item<4-> The performance of the learning agent is measured by the accuracy with which it can classify the arms into $S_{\tau}$ and $S_{\tau}^{c}$ after time horizon $T$. Equivalently, the \emph{loss} $\mathcal{L}(T)$ is defined as
%%\begin{align*}
%%\Ls (T) = \mathbb{I}\big(\lbrace S_{\tau}\cap \hat{S}_{\tau}^{c}\neq \emptyset\rbrace    \cup    \lbrace\hat{S}_{\tau}\cap S_{\tau}^{c}\neq \emptyset\rbrace\big).
%%\end{align*}			
%
%\item<4-> The goal of the learning agent is to minimize the expected loss:
%\begin{align*}
%\Ex[\Ls(T)] &= \Pb\big(\underbrace{\lbrace S_{\tau}\cap \hat{S}_{\tau}^{c} \neq \emptyset \rbrace}_{\textbf{Rejected good arms}}  \cup   \underbrace{\lbrace \hat{S}_{\tau}\cap S_{\tau}^{c} \neq \emptyset\rbrace}_{\textbf{Accepted bad arms}}\big) \\
%%& = 1 - \Pb\big(\lbrace \hat{S}_{\tau}\cap {S}_{\tau}^{c} = \emptyset \rbrace \cap \lbrace \hat{S}_{\tau}^c \cap {S}_{\tau} = \emptyset \rbrace \big )
%\end{align*}
%\end{itemize}
%\end{frame}



\section{Contributions in SMAB}
\begin{frame}
\frametitle{Contributions in SMAB}
\begin{itemize}
\item<1-> We propose the Efficient-UCB-Variance (EUCBV) algorithm for the SMAB setting.
\item<2-> EUCBV takes into account the empirical variances of the arms along with mean estimates to quickly find the optimal arm.
\item<3-> It is the first variance-based arm elimination algorithm for the considered SMAB setting. 
\item<4-> It addresses an open problem discussed in {Auer and Ortner (2010)} of designing an algorithm that can eliminate arms based on variance estimates.
\item<5-> Theoretically it achieves an order-optimal regret bound, the first for an arm elimination algorithm in SMAB setting.
\item<6-> Empirically, it outperforms all the state-of-the-art algorithms for the considered environments.
%\item<5-> We also define a new problem complexity which uses empirical variance estimates along with arm's mean for giving the theoretical bound.
\end{itemize}
\end{frame}


\section{EUCBV Algorithm for SMAB}
% This is LLNCS.DEM the demonstration file of
% the LaTeX macro package from Springer-Verlag
% for Lecture Notes in Computer Science,
% version 2.4 for LaTeX2e as of 16. April 2010
%
\documentclass{llncs}
%
\usepackage{makeidx}  % allows for indexgeneration
\usepackage{macros}
\usepackage{natbib}
%
\begin{document}
%
\frontmatter          % for the preliminaries
%
\pagestyle{headings}  % switches on printing of running heads
\addtocmark{Efficient UCBV} % additional mark in the TOC

\mainmatter              % start of the contributions
%
\title{Efficient UCBV: An Almost Optimal Algorithm using Variance Estimates}
%
\titlerunning{Efficient UCBV}  % abbreviated title (for running head)
%                                     also used for the TOC unless
%                                     \toctitle is used
%
\author{Subhojyoti Mukherjee${}^1$, K. P. Naveen${}^2$, Nandan
Sudarsanam${}^3$, Balaraman Ravindran${}^1$}
%
\authorrunning{Subhojyoti Mukherjee et al.} % abbreviated author list (for running head)
%
%%%% list of authors for the TOC (use if author list has to be modified)
\tocauthor{Subhojyoti Mukherjee, K.P. Naveen, Nandan Sudarsanam and 
 Balaraman Ravindran}
%
\institute{${}^1$Department of Computer Science \& Engineering,\\ ${}^2$Department of Electrical Engineering,
${}^3$Department of Management Studies,\\ Indian Institute of
Technology Madras\\
%\email{subho@cse.iitm.ac.in}
}
%Universit\'{e} de Paris-Sud,
%Laboratoire d'Analyse Num\'{e}rique, B\^{a}timent 425,\\
%F-91405 Orsay Cedex, France
%,\\ WWW home page:
%\texttt{http://users/\homedir iekeland/web/welcome.html}

\maketitle              % typeset the title of the contribution

\begin{abstract}
In this paper, we present a novel algorithm for the stochastic multi-armed bandit (MAB) problem. Our proposed Efficient UCB Variance method, referred to as EUCBV is an arm elimination algorithm based on  UCB-Improved and UCBV strategy which takes into account the empirical variance of the arms and along  with aggressive exploration factors eliminate sub-optimal arms. Through a theoretical analysis, we establish that EUCBV achieves a better gap-dependent regret upper bound than UCB-Improved, MOSS, UCB1 and UCBV algorithms. EUCBV enjoys an order optimal gap-independent regret bound same as that of OCUCB and MOSS and better than UCB-Improved, UCB1 and UCBV. Further, numerical experiments on test-cases with small gaps between optimal and sub-optimal mean rewards show that EUCBV owing to its utilization of the variance estimates of the arms results in lower cumulative regret than several popular UCB variants like MOSS, OCUCB as well as Thompson sampling and Bayes-UCB. 

\keywords{Multi-armed Bandits, Cumulative Regret, UCBV, UCB-Improved}
\end{abstract}

\section{Introduction}
\label{sec:intro}
In this paper we deal with the stochastic multi-armed bandit (MAB) setting. Stochastic MABs are classic instances of sequential learning model where at each timestep a learner is exposed to a finite set of actions (or arms) and it has to choose one arm at a time. After choosing (or pulling) an arm the learner sees the reward for the arm as revealed by the environment. Each of these reward is an i.i.d random variable as sampled from the distribution associated with each arm. The mean of the reward distribution associated with an arm $i$ is denoted by $r_i$ whereas the mean of the reward distribution of the optimal arm $*$ is denoted by $r^*$ such that $r_i < r^*, \forall i\in \A$. One of the fundamental assumptions in stochastic MAB is that the distribution associated with each arm does not change over the entire time horizon $T$. The objective in the stochastic bandit problem is to minimize the cumulative regret, which is defined as follows:
\begin{align*}
R_{T}=r^{*}T - \sum_{i\in A} r_{i}n_{i}(T),
\end{align*}
where $T$ is the number of timesteps, $n_{i}(T)=\sum_{j=1}^T I(I_j=i)$ is the number of times the algorithm has chosen arm $i$ up to timestep $T$.
The expected regret of an algorithm after $T$ timesteps can be written as,
\begin{align*}
\E[R_{T}]= \sum_{i=1}^K \E[n_i (T)] \Delta_i,
\end{align*}
where $\Delta_{i}=r^{*}-r_{i}$ denotes the gap between the means of the optimal arm and the $i$-th arm.

\subsection{Related Works}
\label{sec:related}
%There has been a significant amount of research in the area of stochastic MABs. One of the earliest work can be traced to \cite{thompson1933likelihood}, which deals with  the problem of choosing between two treatments to administer on patients who come in sequentially. Other seminal works include that of  \cite{robbins1952some} and then that of \cite{lai1985asymptotically} which established an asymptotic lower bound for the cumulative regret. It showed that for any consistent allocation strategy, we can have
%$\liminf_{T \to \infty}\frac{\E[R_{T}]}{\log T}\geq\sum_{\{i:r_{i}<r^{*}\}}\frac{(r^{*}-r_{i})}{D(Q_{i}||Q^{*})},$
%where $D(Q_{i}||Q^{*})$ is the Kullback-Leibler divergence between the reward densities $Q_{i}$ and $Q^{*}$, corresponding to arms with mean $r_{i}$ and $r^{*}$, respectively.

	Bandit problems has been extensively studied in several earlier works such as \citet{thompson1933likelihood}, \citet{robbins1952some} and \citet{lai1985asymptotically}. Lai and Robbins in  \citet{lai1985asymptotically} established an asymptotic lower bound for the cumulative regret. Over the years stochastic MABs has seen several algorithms with strong regret guarantees. For further reference an interested reader can look into \citet{bubeck2012regret}. The upper confidence bound algorithms balance the exploration-exploitation dilemma by linking the uncertainty in estimate of an arm with the number of times an arm is pulled, and therefore ensuring sufficient exploration. One of the earliest among these algorithms is UCB1 \citep{auer2002finite}, which has a gap-dependent regret upper bound of  $O\left(\frac{K\log T}{\Delta}\right)$, where $\Delta = \min_{i:\Delta_i>0} \Delta_i$. This result is asymptotically order-optimal for the class of distributions considered. But, the worst case gap-independent regret bound of UCB1 is found to be  $O \left(\sqrt{KT\log T}\right)$. In the later work of \citet{audibert2009minimax}, the authors propose the MOSS algorithm and showed that the worst case gap-independent regret bound of MOSS is $O\left( \sqrt{KT} \right)$ which improves upon UCB1 by a factor of order $\sqrt{\log T}$. However, the gap-dependent regret of MOSS is $O\left( \frac{K^{2}\log\left(T\Delta^{2}/K\right)}{\Delta}\right)$ and in certain regimes, this can be worse than even UCB1 (see \citet{audibert2009minimax,lattimore2015optimally}).
	
	 The UCB-Improved algorithm, proposed in \citet{auer2010ucb}, is a round-based\footnote{An algorithm is \textit{round-based} if it pulls all the arms equal number of times in each round and then eliminates one or more arms that it deems  to be sub-optimal.} variant of UCB1, that 
incurs a gap-dependent regret bound of $O\left(\frac{K\log (T\Delta^{2})}{\Delta}\right)$, which is better than that of UCB1. On the other hand, the worst case gap-independent regret bound of UCB-Improved is $O\left(\sqrt{KT\log K}\right)$. Recently in \citet{lattimore2015optimally}, the authors showed that  the algorithm OCUCB achieves order-optimal gap-dependent regret bound of $O\left(\sum_{i=2}^{K}\frac{\log\left(T/H_i\right)}{\Delta_i}\right)$ where $H_i=\sum_{j=1}^{K}\min\left\lbrace \frac{1}{\Delta_i^2},\frac{1}{\Delta_j^2}\right\rbrace$, and a gap-independent regret bound of $O\left( \sqrt{KT}\right)$. This is the best known gap-dependent and gap-independent regret bounds in the stochastic MAB framework. However, unlike our proposed EUCBV algorithm, OCUCB does not take into account the variance of the arms; as a result, empirically  we find  that our algorithm outperforms OCUCB in all the environments considered. 

	In contrast to the above work, the UCBV \citep{audibert2009exploration} algorithm utilizes variance estimates to compute the confidence intervals for each arm. UCBV has a gap-dependent regret bound of $O\left(\frac{K\sigma_{\max}^{2}\log T}{\Delta}\right)$, where $\sigma_{\max}^{2}$ denotes the maximum variance among all the arms $i\in \A$. Its gap-independent regret bound can be inferred to be same as that of UCB1 i.e $O \left(\sqrt{KT\log T}\right)$. Empirically, \citet{audibert2009exploration} showed that UCBV outperforms UCB1 in several scenarios. 
	
	Another notable design principle which has recently gained a lot of popularity is the Thompson Sampling (TS) algorithm (\citep{thompson1933likelihood}, \citep{agrawal2011analysis})  and  Bayes-UCB (BU) algorithm \citep{kaufmann2012bayesian}. % which employs the Bayesian approach in solving the MAB problem.
The TS algorithm maintains a posterior reward distribution for each arm; at each round, the algorithm samples values from these distribution and the arm corresponding to the highest sample value is chosen. Although TS is found to perform extremely well when the reward distributions are Bernoulli, it is established that with Gaussian priors the worst case regret can be as bad as $\Omega \left( \sqrt{KT\log T}\right)$ \citep{lattimore2015optimally}. The BU algorithm is an extension of the TS algorithm that takes quartile deviations into consideration while choosing arms.
	
	The final design principle we will state is the information theoretic approach of  DMED \citep{honda2010asymptotically} and KLUCB \citep{garivier2011kl} algorithms. The algorithm KLUCB uses Kullbeck-Leibler divergence to compute the upper confidence bound for the arms. KLUCB is stable for a short horizon and is known to reach the \citet{lai1985asymptotically} lower bound in the special case of Bernoulli distribution. However, \citet{garivier2011kl} showed that KLUCB, MOSS and UCB1 algorithms are  empirically outperformed by UCBV in the exponential distribution as they do not take the variance of the arms into consideration. 


\subsection{Contribution}
\label{sec:contri}
In this paper we propose the Efficient UCB Variance (hence referred to as EUCBV) algorithm for the stochastic MAB setting. EUCBV combines the approach of UCB-Improved, CCB \citep{liu2016modification} and UCBV algorithms. EUCBV by virtue of taking into account the empirical variance of the arms performs significantly better than the existing algorithms in the stochastic MAB setting. EUCBV outperforms UCBV \citep{audibert2009exploration} which also takes into account empirical variance but is less powerful than EUCBV because of the usage of exploration regulatory factor and arm elimination parameter by EUCBV. Also we carefully design the confidence interval term with the variance estimates along with the pulls allocated to each arm to balance the risk of eliminating the optimal arm against excessive optimism.   Theoretically we refine the analysis of \citet{auer2010ucb} and prove that for $T\geq K^{2.4}$ our algorithm is order optimal and enjoys a worst case gap-independent regret bound of $O\left( \sqrt{KT} \right)$ same as that of MOSS and OCUCB and better than UCBV, UCB1 and UCB-Improved. Also the gap-dependent regret bound of EUCBV is better than UCB1, UCB-Improved and MOSS but is poorer than OCUCB. But EUCBV gap-dependent bound matches OCUCB in the worst case scenario when all the gaps are equal. Through our theoretical analysis we establish the exact values of the exploration parameters for the best performance of EUCBV. Our proof technique is highly generic and can be easily extended to other MAB settings. An illustrative table containing the bounds is provided in Table \ref{tab:comp-bds}. 

\begin{table}
\caption{Regret upper bound of different algorithms}
\label{tab:comp-bds}
\begin{center}
\begin{tabular}{p{6em}p{12em}p{10em}}
\toprule
Algorithm  & Gap-Dependent & Gap-Independent \\
\hline
EUCBV		& $O\left( \dfrac{K\log (T\Delta^2 /K)}{\Delta}\right)$ & $O\left(\sqrt{KT}\right)$\\
UCB1        & $O\left( \dfrac{K\log T}{\Delta} \right)$ & $O\left(\sqrt{KT\log T}\right)$ \\%\midrule
UCBV        & $O\left( \dfrac{K\sigma_{\max}^{2}\log T}{\Delta} \right)$ & $O\left(\sqrt{KT\log T}\right)$ \\
UCB-Imp 		& $O\left( \dfrac{K\log (T\Delta^2)}{\Delta} \right)$ & $O\left(\sqrt{KT\log K}\right)$ \\%\midrule
MOSS	     	& $O\left( \dfrac{K^2\log (T\Delta^2 /K)}{\Delta}\right)$ & $O\left(\sqrt{KT}\right)$\\%\midrule
OCUCB     	& $O\left( \dfrac{K\log (T/ H_{i})}{\Delta}\right)$ & $O\left(\sqrt{KT}\right)$\\\midrule
\end{tabular}
\end{center}
\vspace*{-2em}
\end{table}

Empirically we show that EUCBV owing to its estimating the variance of the arms performs significantly better than MOSS, OUCUB, UCB-Improved, UCB1, UCBV, Thompson Sampling, Bayes-UCB, DMED, KL-UCB and Median Elimination algorithms. Please note that except UCBV all the afore-mentioned algorithms does not take into account the empirical variance estimates of the arms. Also EUCBV is the first arm-elimination algorithm that takes into account the variance estimates of the arm for minimizing cumulative regret and thereby answers an open question raised by \citet{auer2010ucb}. Also it is first algorithm that follows the same proof technique of UCB-Improved and achieves a gap-independent regret bound of $O\left( \sqrt{KT} \right)$ thereby closing the gap of UCB-Improved \citep{auer2010ucb} which achieved a gap-independent regret bound of $O\left( \sqrt{KT\log K} \right)$. 
	
	The rest of the paper is organized as follows. In section \ref{sec:eucbv} we state the main algorithm EUCBV and in the next section \ref{sec:results} we state all the main results of the paper. In  section \ref{sec:proofTheorem} we establish the proofs of all the Lemma, Theorem and Corollaries and section \ref{sec:expt} contains the numerical experiments. We conclude in section \ref{sec:conc} and discuss about future works.

\section{Algorithm: Efficient UCB Variance}
\label{sec:eucbv}
%%%%%%%%%%%%%%%% alg-custom-block %%%%%%%%%%%%
\algblock{ArmElim}{EndArmElim}
\algnewcommand\algorithmicArmElim{\textbf{\em Arm Elimination}}
 \algnewcommand\algorithmicendArmElim{}
\algrenewtext{ArmElim}[1]{\algorithmicArmElim\ #1}
\algrenewtext{EndArmElim}{\algorithmicendArmElim}

\algblock{ResParam}{EndResParam}
\algnewcommand\algorithmicResParam{\textbf{\em Reset Parameters}}
 \algnewcommand\algorithmicendResParam{}
\algrenewtext{ResParam}[1]{\algorithmicResParam\ #1}
\algrenewtext{EndResParam}{\algorithmicendResParam}

\begin{algorithm}[!h]
\caption{EUCBV}
\label{alg:eucbv}
\begin{algorithmic}
\State {\bf Input:} Time horizon $T$, exploration parameters $\rho$ and $\psi$.
\State {\bf Initialization:} Set $m:=0$, $B_{0}:=A$, $\epsilon_{0}:=1$, $M=\big \lfloor \frac{1}{2}\log_{2} \frac{T}{e}\big\rfloor$, $n_{0}=\bigg\lceil\frac{\log{(\psi T\epsilon_{0}^{2})}}{2\epsilon_{0}}\bigg\rceil$ and  $N_{0}=Kn_{0}$.
\State Pull each arm once
\For{$t=K+1,..,T$}	
\State Pull arm $i\in \argmax_{j\in B_{m}}\bigg\lbrace \hat{r}_{j} + \sqrt{\frac{\rho\hat{v}_{j}\log{(\psi T\epsilon_{m})}}{4 z_{j}}+ \frac{\rho\log{(\psi T\epsilon_{m})}}{4 z_{j}}} \bigg\rbrace$, where $z_j$ is the number of times arm $j$ has been pulled
\State $t:=t+1$
\ArmElim
\State For each arm $i \in B_{m}$, remove arm $i$ from $B_{m}$ if,
\begin{align*}
& \hat{r}_{i} + \sqrt{\frac{\rho\hat{v}_{i}\log{(\psi T\epsilon_{m})}}{4 z_{i}} + \frac{\rho\log{(\psi T\epsilon_{m})}}{4 z_{i}}} < \max_{{j}\in B_{m}}\bigg\lbrace\hat{r}_{j} -\sqrt{\frac{\rho\hat{v}_{j}\log{(\psi T\epsilon_{m})}}{4 z_{j}} + \frac{\rho\log{(\psi T\epsilon_{m})}}{4 z_{j}}} \bigg\rbrace
\end{align*}
%\State $|B_{m}|:=|B_{m}|-1$
\EndArmElim

\If{$t\geq N_{m}$ and $m\leq M$}
\ResParam
\State $\epsilon_{m+1}:=\frac{\epsilon_{m}}{2}$\vspace{0.5ex}
\State $B_{m+1}:=B_{m}$
\State $n_{m+1}:=\bigg\lceil\frac{\log{(\psi T\epsilon_{m+1}^{2})}}{2\epsilon_{m+1}}\bigg\rceil$
\State $N_{m+1}:=t+|B_{m+1}| n_{m+1}$
\State $m:=m+1$
\EndResParam
\State Stop if $|B_{m}|=1$ and pull ${i}\in B_{m}$ till $T$ is reached.
\EndIf
\EndFor
\end{algorithmic}
%\vspace*{-0.42em}
\end{algorithm}
%\vspace*{-0.42em}
\textbf{2.1 Notations:} We denote the set of arms by $\A$, with the individual arms labeled $i, i=1,\ldots,K$. We denote an arbitrary round of EUCBV by $m$. For simplicity, we assume that the optimal arm is unique and denote it by ${*}$. We denote the sample mean of the rewards for an arm $i$ at time instant $t$ by $\hat{r}_{i}(t)=\frac{1}{z_{i}(t)}\sum_{\ell=1}^{z_i(t)} X_{i,\ell}$, where $X_{i,\ell}$ is the reward sample received when arm $i$ is pulled for the $z$-th time. $z_i(t)$ is the number of times an arm $i$ has been pulled till timestep $t$. We denote the true variance of an arm by $\sigma_i^{2}$ while $\hat{v}_{i}(t)$ is the estimated variance, i.e., $\hat{v}_{i}(t)=\frac{1}{z_i(t)}\sum_{\ell=1}^{z_{i}(t)}(X_{i,\ell}-\hat{r}_{i})^{2}$. Whenever there is no ambiguity about the underlaying  time index $t$, for simplicity we neglect $t$ from the notations and simply use  $\hat{r}_i, \hat{v}_i,$ and $z_i$ to denote the respective quantities. We assume the rewards of all arms are bounded in $[0,1]$.

\textbf{2.2 The algorithm:} Earlier arm elimination algorithms like Median Elimination \citep{even2006action} and UCB-Improved \citep{auer2010ucb} mainly suffered from two basic problems: \\
\begin{inparaenum}[\bfseries(i)]
\item \textit{Initial exploration:} Both of these algorithms pull each arm equal number of times in each round, and hence waste a significant number of pulls in initial explorations. \\
\item \textit{Conservative arm-elimination:} In UCB-Improved, arms are eliminated conservatively, i.e, only after $\epsilon_{m}<\frac{\Delta_{i}}{2}$, the sub-optimal arm $i$ is discarded with high probability. The quantity $\epsilon_{m}$ is initialized to $1$ and halved after every round. In the worst case scenario when $K$ is large and the gaps are uniform  ($r_{1}=r_{2}=\cdots=r_{K-1}<r^{*}$) and small this results in very high regret.\\
\end{inparaenum}
%For any round $m$ UCB-Improved pulls all arms $n_{m}=\left\lceil \frac{ 2\log(T\epsilon_{m})}{\epsilon_{m}} \right\rceil$ number of times. The quantity $\epsilon_{m}$ is initialized to $1$ and halved after every round.
\\
	EUCBV algorithm which is mainly based on the arm elimination technique of the UCB-Improved algorithm remedies these by employing exploration regulatory factor $\psi$ and arm elimination parameter $\rho$ for aggressive elimination of sub-optimal arms. Along with these, like CCB \citep{liu2016modification} algorithm, EUCBV uses optimistic greedy sampling whereby at every timestep it only pulls the arm with the highest upper confidence bound rather than pulling all the arms equal number of times in each round. Also, unlike the UCB-Improved, UCB1. MOSS and OCUCB algorithm (which are based on mean estimation) EUCBV employs mean and variance estimates (as in \citet{audibert2009exploration}) for arm elimination. Further, we allow for arm-elimination at every time-step, which is in contrast to the earlier work (e.g., \citet{auer2010ucb}; \citet{even2006action}) where the arm elimination takes place only at the end of the respective exploration rounds. 






\section{Main Results}
\label{sec:results}
\subsection{Lemma 1}

A technical lemma used to prove Theorem \ref{Result:Theorem:1} is presented below.

\begin{lemma}
\label{results:Lemma:1}
If $T\geq K^{2.7}$, $\psi=\dfrac{T}{ K^2}$, $\rho=\dfrac{1}{2}$ and $m\leq \dfrac{1}{2} \log_2(\dfrac{T}{e}) $, then,
\begin{align*}
\dfrac{\rho m \log(2)}{\log(\psi T) - 2m\log( 2)} \leq 1
\end{align*}
\end{lemma}

\begin{proof}
The proof is given in Section \ref{sec:proofTheorem:Lemma1}.
\end{proof}

We present below the main theorem of the paper which establishes the regret upper bound for the EUCBV  algorithm. 

\subsection{Main Theorem}
\begin{theorem}
\label{Result:Theorem:1}
For $T\geq K^{2.7}$, the regret $R_T$ for EUCBV satisfies
\begin{align*}
 \E [R_{T}] \leq &\sum\limits_{i\in \A :\Delta_{i} > b}\bigg\lbrace\bigg(\dfrac{C_{1}(\rho)T^{1-\rho}}{\Delta_{i}^{2\rho -1}}\bigg) + \bigg(\Delta_{i}+\dfrac{41\log{(\psi  T\Delta_{i}^{4})}}{\Delta_{i}}\bigg) + \bigg(\dfrac{ C_{2}(\rho) T^{1-\rho}}{\Delta_{i}^{2\rho-1}} \bigg)\bigg \rbrace \\ 
 %%%%%%%%%%%%%%%%
  & +\sum\limits_{i\in \A :0 < \Delta_{i}\leq b}\bigg(\dfrac{C_{2}(\rho)T^{1-\rho}}{b^{2\rho -1}} \bigg) + \max_{i\in \A :0 < \Delta_{i}\leq b}\Delta_{i}T
\end{align*}
for all $b\geq\sqrt{\frac{e}{T}}$. In the above, $C_1(x) = \frac{2^{2+x}.9^{x}}{\psi^{x}}$ and $C_2(x) = \frac{2^{\frac{\rho}{2}+\frac{9}{4}}.3^{x+\frac{1}{2}}}{\psi^{x}}$.
\end{theorem}

\begin{proof}
The proof comprises of three modules. In the first module we prove the necessary conditions for arm elimination within a specified number of rounds, which is motivated by the technique in \cite{auer2010ucb}. In this module we combine the approach of \cite{audibert2009exploration} with that of  \cite{auer2010ucb} while breaking down the confidence interval term which contains the estimated variance   term for an arm. Please note that even though \cite{audibert2009exploration} uses Bernstein inequality to obtain the  bound, we use Chernoff-Hoeffding bound. This is because of our choice of $\rho$ which cannot be decreased below $\frac{1}{2}$ as it may lead to a regret polynomial in $T$ and usage of Bernstein inequality will force the $\rho$ to take a value lower than $\frac{1}{2}$. The second module bounds the number of pulls required if an arm is eliminated on or before a particular number of rounds. Note that the number of pulls allocated in a round $m$ for each arm is atmost $n_{m}:=\bigg\lceil\frac{\log{(\psi T\epsilon_{m}^{2})}}{2\epsilon_{m}}\bigg\rceil$ which is much lower than the pulls of each arm required by UCB-Improved or Median-Elimination. The third module deals with bounding the regret given a sub-optimal arm eliminates the optimal arm. The detailed proof is given in Section \ref{sec:proofTheorem:Theorem1}.
\end{proof}
Next, we specialize the result of Theorem \ref{Result:Theorem:1} in Corollary \ref{Result:Corollary:1} and Corollary \ref{Result:Corollary:2}.

\subsection{Corollary 1}
\begin{corollary}[\textbf{\textit{Gap-dependent bound}}]
\label{Result:Corollary:1}
With $\psi=\frac{T}{K^2}$ and $\rho=\frac{1}{2}$, we have the following gap-dependent bound for the regret of EUCBV:
\begin{align*}
\E [R_T] & \sum_{i\in \A:\Delta_{i} > b}\bigg\lbrace 34 K + \dfrac{82\log{(\dfrac{T\Delta_{i}^{2}}{ K})}}{\Delta_{i}}\bigg\rbrace 
 + \max\limits_{i\in \A:\Delta_{i}\leq b}\Delta_{i}T 
	\end{align*} 
\end{corollary}
\begin{proof}
The proof is given in Section \ref{sec:proofTheorem:Corollary1}.
\end{proof}
Thus, we clearly see that the most significant term in the gap-dependent bound is $\dfrac{82K\log{(T\Delta^{2}/K)}}{\Delta}$ and it is better than UCB1, UCBV, MOSS and UCB-Improved. In \citet{audibert2010best} the authors define the term $H_1=\sum_{i=1}^{K}\frac{1}{\Delta_i^2}$ as the hardness of a problem and in \citet{bubeck2012regret} the authors conjectured that the gap-dependent regret upper bound can match the quantity of $O\left(\dfrac{K\log{(T/H_1)}}{\Delta}\right)$. But \citet{lattimore2015optimally} proved that the gap-dependent regret bound cannot be lower than $O\left(\sum_{i=2}^{K}\frac{\log\left(T/H_i\right)}{\Delta_i}\right)$, where $H_i=\sum_{j=1}^{K}\min\lbrace \frac{1}{\Delta_i^2},\frac{1}{\Delta_j^2}\rbrace$ and only in the worst case scenario, when all the gaps are equal then $H_1=H_{i}=\sum_{i=1}^{K}\frac{1}{\Delta^2}$. In such a scenario the EUCBV gap-dependent bound of $O\left(\dfrac{K\log{(T\Delta^{2}/ K)}}{\Delta}\right)$ reduces to $O\left(\dfrac{K\log{(T/H_1)}}{\Delta}\right)$ and hence matches the gap-dependent bound of OCUCB.

\subsection{Corollary 2}

\begin{corollary}[\textbf{\textit{Gap-independent bound}}]
\label{Result:Corollary:2}
With $\psi=\frac{T}{K^2}$ and $\rho=\frac{1}{2}$, we have the following gap-independent bound for the regret of EUCBV:
\begin{align*}
\E[R_{T}] & \leq 51 K^2 + 82\sqrt{KT}
	\end{align*} 
\end{corollary}
\begin{proof}
The proof is given in Section \ref{sec:proofTheorem:Corollary2}.
\end{proof}
Here, in the gap-independent bound of EUCBV the most significant term is $O\left(\sqrt{KT}\right)$ which exactly matches the upper bound of MOSS and OCUCB and is better than UCB-Improved, UCB1 and UCBV.

\section{Proofs}
\label{sec:proofTheorem}
%\subsection{Lemma 1}
%\label{sec:proofTheorem:Lemma1}
We first present a few technical lemmas that is required  to prove the result in Theorem \ref{Result:Theorem:1}.

\begin{lemma}
\label{proofTheorem:Lemma:1}
If $T\geq K^{2.4}$, $\psi=\frac{T}{ K^2}$, $\rho=\frac{1}{2}$ and $m\leq \frac{1}{2} \log_2\left(\frac{T}{e}\right) $, then,
\begin{align*}
\dfrac{\rho m \log(2)}{\log(\psi T) - 2m\log( 2)} \leq \frac{3}{2}.
\end{align*}
\end{lemma}



\begin{lemma}
\label{proofTheorem:Lemma:2}
If $T\geq K^{2.4}$, $\psi=\frac{T}{ K^2}$, $\rho =\frac{1}{2}$, $m_i = min\lbrace m|\sqrt{4\epsilon_{m} } < \frac{\Delta_i}{4} \rbrace $ and $c_{i} =\sqrt{\frac{\rho (\hat{v}_i + 2)\log (\psi T\epsilon_{m_{i}})}{4 z_i}}$, then,
%\begin{align*}
$c_{i} < \frac{\Delta_i}{4}$.
%\end{align*}
\end{lemma}



\begin{lemma}
\label{proofTheorem:Lemma:3}
If $m_i = min\lbrace m|\sqrt{4\epsilon_{m} } < \frac{\Delta_i}{4} \rbrace $,  $\bar{c}_i=\sqrt{\frac{\rho (\sigma_{i}^{2}+\sqrt{\epsilon_{m_{i}}} + 2)\log(\psi T\epsilon_{m_{i}})}{4z_i}}$ and $n_{m_i} = \frac{\log{(\psi T\epsilon_{m_{i}})}}{2\epsilon_{m_{i}}}$ then we can show that,
\begin{align*}
\mathbb{P}\left( \hat{r}_{i} > r_{i}+ \bar{c}_i\right) 
+ \mathbb{P}\left( \hat{v}_{i}\geq \sigma_{i}^{2}+\sqrt{\epsilon_{m_{i}}}\right) \leq \dfrac{2}{(\psi  T\epsilon_{m_{i}})^{\frac{3\rho}{2}}}.
\end{align*}
\end{lemma}



\begin{lemma}
\label{proofTheorem:Lemma:4}
If $m_i = min\lbrace m|\sqrt{4\epsilon_{m} } < \frac{\Delta_i}{4} \rbrace $, $\psi=\frac{T}{ K^2}$, $\rho=\frac{1}{2}$, $c_{i} =\sqrt{\frac{\rho(\hat{v}_i + 2)\log (\psi T\epsilon_{m_{i}})}{4 z_{i}}}$ and $n_{m_i}=\frac{\log{(\psi T\epsilon_{m_{i}}^{2})}}{2\epsilon_{m_{i}}}$ then in the $m_i$-th round, 
\begin{align*}
\Pb\lbrace c^{*} > c_i \rbrace  \leq \dfrac{182 K^4}{T^{\frac{5}{4}}\sqrt{\epsilon_{m_i}}}.
\end{align*}
\end{lemma}



\begin{lemma}
\label{proofTheorem:Lemma:5}
If $m_i = min\lbrace m|\sqrt{4\epsilon_{m} } < \frac{\Delta_i}{4} \rbrace $,$\psi=\frac{T}{ K^2}$, $\rho=\frac{1}{2}$, $c_{i} =\sqrt{\frac{\rho (\hat{v}_i + 2)\log (\psi T\epsilon_{m_{i}})}{4 z_i}}$ and $n_{m_i}=\frac{\log{(\psi T\epsilon_{m_{i}}^{2})}}{2\epsilon_{m_{i}}}$ then in the $m_i$-th round, 
\begin{align*}
\Pb\lbrace z_i < n_{m_i} \rbrace  \leq \dfrac{182 K^4}{T^{\frac{5}{4}}\sqrt{\epsilon_{m_i}}}.
\end{align*}
\end{lemma}



%\begin{lemma}
%\label{proofTheorem:Lemma:6}
%For $T\geq K^{2.4}$, $\epsilon_{m_i}\geq \sqrt{\frac{e}{T}}$, $\psi=\frac{T}{K^2}$ and $\rho=\frac{1}{2}$,  
%\begin{align*}
%\dfrac{6K}{(\psi T \epsilon_{m_i})^{\frac{3\rho}{2}}} > \dfrac{K\log T}{(\psi T)^{3\rho}}\sum_{m=0}^{m_i}\dfrac{1}{\epsilon_{m_i}^{3\rho + 1}}
%\end{align*}
%\end{lemma}



%\begin{lemma}
%\label{proofTheorem:Lemma:6}
%For all bounded rewards in $[0,1]$, $\frac{\Delta_i}{4} \geq \frac{\Delta_i}{4\sigma_i^2 + 4} $.
%\end{lemma}



\begin{lemma}
\label{proofTheorem:Lemma:6}
For two integer constants $c_1$ and $c_2$, if $20 c_1 \leq c_2$ then,
\begin{align*}
c_1 \frac{4\sigma_i^2 + 4}{\Delta_i}\log\bigg( \frac{T\Delta_i^2}{K}\bigg) \leq c_2 \frac{\sigma_i^2}{\Delta_i}\log\bigg( \frac{T\Delta_i^2}{K}\bigg).
\end{align*}
\end{lemma}


%\begin{lemma}
%\label{proofTheorem:Lemma:8}
%If $m_*$ be the first round that the optimal arm $*$ gets eliminated, then we can show that the regret is upper bounded by,
%
%\begin{align*}
%\sum_{m_{*}=0}^{max_{j\in \A^{'}}m_{j}}\sum_{i\in \A^{''}:m_{i}>m_{*}}\bigg(\dfrac{388 K}{(\psi  T\epsilon_{m_{*}})^{\frac{3\rho}{2}}} \bigg).T\max_{j\in \A^{''}:m_{j}\geq m_{*}}{\Delta}_{j} \\
%%%%%%%%%%%%%%%%%%%%%%%%%
% \leq\sum_{i\in \A^{'}}\dfrac{C_2^{'} K^{\frac{5}{2}}}{\sqrt{T\Delta_i}} +\sum_{i\in \A^{''}\setminus \A^{'}}\dfrac{C_2^{'} K^{\frac{5}{2}}}{\sqrt{T b}}
%\end{align*}
%
%\end{lemma}


The proofs of lemmas \ref{proofTheorem:Lemma:1} - \ref{proofTheorem:Lemma:6} can be found in Appendix ~\ref{App:Lemma:1}, ~\ref{App:Lemma:2}, ~\ref{App:Lemma:3}, ~\ref{App:Lemma:4}, ~\ref{App:Lemma:5} and
 ~\ref{App:Lemma:6} respectively.

%The proofs of all the Lemmas can be found in Appendix ~\ref{App:Lemma:1} - Appendix ~\ref{App:Lemma:9} respectively.

\subsection*{Proof of Theorem 1}
\label{sec:proofTheorem:Theorem1}
\begin{customproof}{1}
For each sub-optimal arm ${i}\in\mathcal{A}$, let $m_{i}=\min{\left\lbrace m|\sqrt{4\epsilon_{m_i}} < \frac{\Delta_{i}}{4}\right\rbrace}$. Also, let $\A^{'}=\lbrace i\in \A: \Delta_{i} > b \rbrace$ and $\A^{''}=\lbrace i\in \A: \Delta_{i} > 0 \rbrace$. Note that as all rewards are bounded in $[0,1]$, it implies that $0\leq \sigma_i^2 \leq \frac{1}{4},\forall i\in \A$. Now, as in \citet{auer2010ucb}, we bound the regret under the following two cases: 
\begin{itemize}
\item {Case $(a)$}: some sub-optimal arm ${i}$ is not eliminated in round $m_{i}$ or before and the optimal arm ${*}\in B_{m_{i}}$
\item {Case $(b)$}: an arm ${i}\in B_{m_i}$ is eliminated in round $m_{i}$ (or before), or there is no optimal arm $*\in B_{m_i}$
\end{itemize} 
The details of each case are contained in the following sub-sections.

%Note that in in round $m_i$ as $\sqrt{4\epsilon_{m_i}} < \dfrac{\Delta_{i}}{4}$ implies that $\sqrt{4\epsilon_{m_i}} < \dfrac{\Delta_{i}}{4\sigma_i^2}$, since $\sigma_i^2\in (0,1]$

\textbf{Case $(a)$:}
For simplicity, let $c_{i} := \sqrt{\frac{\rho (\hat{v}_i + 2) \log (\psi T\epsilon_{m_{i}})}{4 z_{i}}}$ denote the length of the confidence interval corresponding to arm $i$ in round $m_i$. Thus, in round $m_i$ (or before) whenever $z_i \geq n_{m_{i}}\ge\frac{\log{(\psi T\epsilon_{m_{i}}^{2})}}{2\epsilon_{m_{i}}}$, by applying Lemma \ref{proofTheorem:Lemma:2} we obtain $c_{i} < \frac{\Delta_{i}}{4}$.
%\begin{align*}
%	c_{i} < \dfrac{\Delta_{i}}{4} 
%\end{align*}
Now, the sufficient conditions for arm $i$ to get eliminated by an optimal arm in round $m_i$ is given by
	\begin{eqnarray}
	\hat{r}_{i} \leq r_{i} + c_{i} \text{, } \label{eq:armelim-casea}
 	\hat{r}^{*} \geq r^{*} - c^{*} \text{, } c_{i} \geq c^* \text{ and } z_i \geq n_{m_i} .
	\end{eqnarray}

Indeed, in round $m_i$ suppose (\ref{eq:armelim-casea}) holds, then we have
%	 
  \begin{align*}
\hat{r}_{i} + c_{i}&\leq r_{i} + 2c_{i} 
= r_{i} + 4c_{i} - 2c_{i} \\
 &< r_{i} + \Delta_{i} - 2c_{i}
 \leq r^{*} -2c^{*} 
 \leq \hat{r}^{*} - c^{*}
  \end{align*}
  so that a sub-optimal arm ${i} \in \A^{'}$ gets eliminated.	
Thus, the probability of the complementary event of these four conditions yields a bound on the probability that arm $i$ is not eliminated in round $m_i$. A bound on the complementary of the first condition is given by,

\begin{align}
\mathbb{P}(\hat{r}_{i}> r_{i} + c_{i})
&\leq \mathbb{P}\left( \hat{r}_{i} > r_{i}+ \bar{c}_i\right) 
+ \mathbb{P}\left( \hat{v}_{i}\geq \sigma_{i}^{2}+\sqrt{\epsilon_{m_{i}}}\right)\label{eq:prob_eq2}
\end{align}
where 
\begin{align*}
\bar{c}_i=\sqrt{\dfrac{\rho (\sigma_{i}^{2}+\sqrt{\epsilon_{m_{i}}} + 2)\log(\psi T\epsilon_{m_{i}})}{4n_{m_i}}}.
\end{align*}

%%%%%%%%%%%%%%%%%%%%%%%%%%%%%%%%%%
% Shifted as Lemma
%%%%%%%%%%%%%%%%%%%%%%%%%%%%%%%%%%
%Note that, substituting $ n_{m_i} \geq \frac{\log{(\psi T\epsilon_{m_{i}})}}{2\epsilon_{m_{i}}}$, $\bar{c}_i$ can be simplified to obtain,
%\begin{align}
%\bar{c}_i
%\leq \sqrt{\dfrac{\rho\epsilon_{m_{i}}(\sigma_{i}^{2}+\sqrt{\epsilon_{m_{i}}} + 2)}{2}}\leq \sqrt{ \epsilon_{m_{i}}}.
%\label{si_bar_equn}
%\end{align}
%%
%The first term in the LHS of (\ref{eq:prob_eq2}) can be bounded using the Bernstein inequality as below:
%\begin{align}
%&\mathbb{P}\left( \hat{r}_{i} > r_{i}+ \bar{c}_i\right)\nonumber 
%\le \exp\left(- \dfrac{(\bar{c}_i)^2 z_{i}}{2\sigma_i^2 + \frac{2}{3}\bar{c}_i} \right)\nonumber 
%%%%%%%%%%%%%%%%
%\\
%& \overset{(a)}{\le} \exp\left(- \rho \left(\dfrac{3\sigma_{i}^{2}+3\sqrt{\epsilon_{m_{i}}} + 6}{6\sigma_i^2 + 2\sqrt{\epsilon_{m_i}}} \right)\log(\psi  T\epsilon_{m_{i}}\right)\nonumber \\
%%%%%%%%%%%%%%%%
%% &\le \exp\left(- \rho (\sigma_{i}^{2}+\sqrt{\epsilon_{m_{i}}} + 2)\log(\psi  T\epsilon_{m_{i}})\right)\nonumber \\
%%%%%%%%%%%%%%%%
%& \overset{(b)}{\leq} \exp\left(- \rho \log(\psi  T\epsilon_{m_{i}})\right) 
%%%%%%%%%%%%%%%%
%\le \dfrac{1}{(\psi  T\epsilon_{m_{i}})^{\rho}}
%\label{lhs1_equn}
%\end{align}
%where, $(a)$ is obtained by substituting equation \ref{si_bar_equn} and $(b)$ occurs because for all $\sigma_{i}^2 \in [0,1]$, $\left(\dfrac{3\sigma_{i}^{2}+3\sqrt{\epsilon_{m_{i}}} + 6}{6\sigma_i^2 + 2\sqrt{\epsilon_{m_i}}}\right) \geq 1$ .
%
% 
%The second term in the LHS of (\ref{eq:prob_eq2}) can be simplified as follows:
%\begin{align}
%&\mathbb{P}\bigg\lbrace \hat{v}_{i}\geq \sigma_{i}^{2}+\sqrt{\epsilon_{m_{i}}}\bigg\rbrace\nonumber\\
%%%%%%%%%%%%%%%%%%%
%&\leq \mathbb{P}\bigg\lbrace \dfrac{1}{n_{i}}\sum_{t=1}^{n_{i}}(X_{i,t}-r_{i})^{2}-(\hat{r}_{i}-r_{i})^{2}\geq \sigma_{i}^{2}+\sqrt{\epsilon_{m_{i}}}\bigg\rbrace\nonumber\\
%%%%%%%%%%%%%%%%%%%
%&\leq \mathbb{P}\bigg\lbrace \dfrac{\sum_{t=1}^{n_{i}}(X_{i,t}-r_{i})^{2}}{n_{i}}\geq \sigma_{i}^{2}+\sqrt{\epsilon_{m_{i}}} \bigg\rbrace\nonumber\\
%%%%%%%%%%%%%%%%%%%
%&\overset{(a)}{\leq} \mathbb{P}\bigg\lbrace \dfrac{\sum_{t=1}^{n_{i}}(X_{i,t}-r_{i})^{2}}{n_{i}}\geq \sigma_{i}^{2} + \bar{c}_i\bigg\rbrace \nonumber\\
%%%%%%%%%%%%%%%%%%%
%&\overset{(b)}{\leq} \exp\left(- \rho \left(\dfrac{3\sigma_{i}^{2}+3\sqrt{\epsilon_{m_{i}}} + 6}{6\sigma_i^2 + 2\sqrt{\epsilon_{m_i}}} \right)\log(\psi  T\epsilon_{m_{i}})\right)
%%%%%%%%%%%%%%%%%%
%\le \dfrac{1}{(\psi  T\epsilon_{m_{i}})^{\rho}}
%\label{lhs2_equn}
%\end{align}
%where inequality $(a)$ is obtained using (\ref{si_bar_equn}), while $(b)$ follows from the Bernstein inequality.
  
%Thus, using (\ref{lhs1_equn}) and (\ref{lhs2_equn}) in (\ref{eq:prob_eq2}) we obtain $\mathbb{P}(\hat{r}_{i}> r_{i} + c_{i})\le \dfrac{2}{(\psi  T\epsilon_{m_{i}})^{\rho}}$. 

From Lemma \ref{proofTheorem:Lemma:3} we can show that $\mathbb{P}\left( \hat{r}_{i} > r_{i}+ \bar{c}_i\right) + \mathbb{P}\left( \hat{v}_{i}\geq \sigma_{i}^{2}+\sqrt{\epsilon_{m_{i}}}\right) \leq \dfrac{2}{(\psi  T\epsilon_{m_{i}})^{\frac{3\rho}{2}}}$. Similarly, $\mathbb{P}\lbrace\hat{r}^{*} < r^{*} - c^{*}\rbrace \leq \dfrac{2}{(\psi  T\epsilon_{m_{i}})^{\frac{3\rho}{2}}}$. Summing the above two contributions, the probability that a sub-optimal arm ${i}$ is not eliminated on or before $m_{i}$-th round by the first two conditions is,  
\begin{eqnarray}
\bigg(\dfrac{4}{(\psi T\epsilon_{m_{i}})^{\frac{3\rho}{2}}} \bigg) \label{eq:arm:elim:c1}
\end{eqnarray}
 

Again, from Lemma \ref{proofTheorem:Lemma:4} and Lemma \ref{proofTheorem:Lemma:5} we can bound the probability of the  complementary of the event $c_{i} \geq c^* $ and $ z_i \geq n_{m_i}$ by,

\begin{eqnarray}
\dfrac{182 K^4}{T^{\frac{5}{4}}\sqrt{\epsilon_{m_i}}} + \dfrac{182 K^4}{T^{\frac{5}{4}}\sqrt{\epsilon_{m_i}}}\leq \dfrac{364 K^4}{T^{\frac{5}{4}}\sqrt{\epsilon_{m_i}}} \label{eq:arm:elim:c2}.
\end{eqnarray}

Also, for $(\ref{eq:arm:elim:c1})$ we can show that for any $\epsilon_{m_i}\in[\sqrt{\frac{e}{T}},1]$
\begin{align*}
\bigg(\dfrac{4}{(\psi T\epsilon_{m_{i}})^{\frac{3\rho}{2}}} \bigg) &\overset{(a)}{\leq} \bigg(\dfrac{4}{(\frac{T^2}{K^2}\epsilon_{m_{i}})^{\frac{3}{4}}} \bigg)\leq \bigg(\dfrac{4 K^{\frac{3}{2}}}{(T^\frac{3}{2} \epsilon_{m_i}^{\frac{1}{4}}\sqrt{\epsilon_{m_{i}}})}\bigg)\\
%%%%%%%%%%%%%%%%%%%%%%%
 &\overset{(b)}{\leq} \bigg(\dfrac{4 K^{\frac{3}{2}}}{(T^{\frac{3}{2}-\frac{1}{8}}\sqrt{\epsilon_{m_{i}}})}  \bigg)
\leq \dfrac{4 K^4}{T^{\frac{5}{4}}\sqrt{\epsilon_{m_i}}}.
\end{align*}

Here, in $(a)$ we substitute the values of $\psi$ and $\rho$ and $(b)$ follows from the identity $\epsilon_{m_i}^{\frac{1}{4}}\geq (\frac{e}{T})^{\frac{1}{8}} $ as $\epsilon_{m_i}\geq \sqrt{\frac{e}{T}}$.

Summing up over all arms in $\A^{'}$ and bounding the regret for all the \textit{four} arm elimination conditions ( $(\ref{eq:arm:elim:c1}) + (\ref{eq:arm:elim:c2})$)for each arm $i\in \A^{'}$ trivially by $T\Delta_{i}$, we obtain
	\begin{align*}
&\sum_{i\in \A^{'}}\bigg(\dfrac{4 K^4 T\Delta_i}{T^{\frac{5}{4}}\sqrt{\epsilon_{m_i}}}\bigg) + \sum_{i\in \A^{'}}\bigg(\dfrac{364 K^4 T\Delta_i}{T^{\frac{5}{4}}\sqrt{\epsilon_{m_i}}}\bigg)\\
%%%%%%%%%%%%%%%%%%%%%%%%%%%%%
&\overset{(a)}{\leq}\sum_{i\in \A^{'}}\bigg(\dfrac{368 K^4 T\Delta_{i}}{T^{\frac{5}{4}}\left(\frac{\Delta_{i}^{2}}{4.16}\right)^{\frac{1}{2}}}\bigg)
%%%%%%%%%%%%%%%%%%%%%%%%%%%%%%%
\overset{(b)}{\leq} \sum_{i\in \A^{'}}\bigg(\dfrac{C_1 K^4}{(T)^{\frac{1}{4}}}\bigg).\\  
%%%%%%%%%%%%%%%%%%%%%%%%%%%%%%%
	\end{align*}

%   \begin{align*}
%&\sum_{i\in \A^{'}}\bigg(\dfrac{388 K T\Delta_{i}}{(\psi T\epsilon_{m_{i}})^{\frac{3\rho}{2}}}\bigg)
%\leq\sum_{i\in \A^{'}}\bigg(\dfrac{388 K T\Delta_{i}}{(\psi T\dfrac{\Delta_{i}^{2}}{4.16})^{\frac{3\rho}{2}}}\bigg)\\
%%%%%%%%%%%%%%%%%%%%%%%%%%%%%%%
%&\leq \sum_{i\in \A^{'}}\bigg(\dfrac{388.2^{2+2\frac{3\rho}{2}}.16^{\frac{3\rho}{2}} K T^{1-\frac{3\rho}{2}}}{\psi^{\frac{3\rho}{2}}\Delta_{i}^{2\frac{3\rho}{2} -1}}\bigg)\\  
%%%%%%%%%%%%%%%%%%%%%%%%%%%%%%%
%& \overset{(a)}{\leq} \sum_{i\in \A^{'}}\bigg(\dfrac{388.2^{2+\frac{3}{2}}.16^{\frac{3}{4}} K T^{1-\frac{3}{4}}}{(\frac{T}{K^2})^{\frac{3}{4}}\Delta_{i}^{2.\frac{3}{4} -1}}\bigg)\leq \sum_{i\in \A^{'}}\dfrac{C_1 K^{\frac{5}{2}}}{\sqrt{T\Delta_i}}  
%   \end{align*}
%Here in $(a)$ we substitute the values of $\rho$ and $\psi$ and $C_1$ denotes a constant integer value.\\
Here, $(a)$ happens because $\sqrt{4\epsilon_{m_i}} < \frac{\Delta_i}{4}$, and in $(b)$, $C_1$ denotes a constant integer value.\\


%%%%%%%%%%%%%%%%%%%%%%%%%%%%%%%%%%%%%
% Case (b)
%%%%%%%%%%%%%%%%%%%%%%%%%%%%%%%%%%%%%
\textbf{Case $(b)$:} Here, there are two sub-cases to be considered.
% \subsection*{Case $b$: \textit{An arm ${i}\in B_{m_i}$ is eliminated in round $m_{i}$ or before or there is no $*\in B_{m_i}$}}

\noindent
\textbf{Case $(b1)$ (\textit{${*}\in B_{m_{i}}$ and each ${i}\in \A^{'}$ is  eliminated on or before $m_{i}$ }): } Since we are eliminating a sub-optimal arm ${i}$ on or before round $m_{i}$, it is pulled no longer than, 
 \begin{align*}
 z_{i} < \bigg\lceil\dfrac{\log{(\psi T\epsilon_{m_{i}}^{2})}}{2\epsilon_{m_{i}}}\bigg\rceil
 \end{align*}
%\hspace*{4em}
%%$, since $\sqrt{\rho_{a}\epsilon_{m_{i}}}\leq\dfrac{\Delta_{i}}{2}
So, the total contribution of ${i}$  till round $m_{i}$ is given by, 
\begin{align*}
&\Delta_{i}\bigg\lceil\dfrac{\log{(\psi T\epsilon_{m_{i}}^{2})}}{2\epsilon_{m_{i}}}\bigg\rceil
\overset{(a)}{\leq}    \Delta_{i}\bigg\lceil\dfrac{\log{(\psi T(\dfrac{\Delta_{i}}{16 \times 256})^{4})}}{2(\dfrac{\Delta_{i}}{4\sqrt{4}})^{2}}\bigg\rceil \\
%%%%%%%%%%%%%%%%%%%%%%%%%%%%%%
&\leq   \Delta_{i}\bigg(1+\dfrac{32\log{(\psi T(\dfrac{\Delta_{i}^{4}}{16384})}}{\Delta_{i}^{2}}\bigg)
\leq \Delta_{i}\bigg(1+\dfrac{32\log{(\psi T\Delta_{i}^{4})}}{\Delta_{i}^{2}}\bigg) .
\end{align*} 

Here, $(a)$ happens because $\sqrt{4\epsilon_{m_{i}}} < \frac{\Delta_{i}}{4}$. Summing over all arms in $\A^{'}$ the total regret is given by, 
\begin{align*}
&\sum_{i\in \A^{'}}\Delta_{i}\bigg(1+\dfrac{32\log{(\psi T\Delta_{i}^{4}})}{\Delta_{i}^{2}}\bigg) = \sum_{i\in \A^{'}}\bigg(\Delta_{i} +\dfrac{32\log{(\psi T\Delta_{i}^{4}})}{\Delta_{i}}\bigg) \\
%%%%%%%%%%%%%%%%%%%%%%%%%%%
&\overset{(a)}{\leq} \sum_{i\in \A^{'}} \left(\Delta_{i}+\dfrac{64\log{( \frac{T\Delta_{i}^{2}}{K})}}{\Delta_{i}}\right)\\
%%%%%%%%%%%%%%%%%%%%%%%%%%%
&\overset{(b)}{\leq} \sum_{i\in \A^{'}} \left(\Delta_{i} +\dfrac{16(4\sigma_i^2 + 4)\log{( \frac{T\Delta_{i}^{2}}{K})}}{\Delta_{i}}\right)\\
&%%%%%%%%%%%%%%%%%%%%%%%%%%%
\overset{(c)}{\leq} \sum_{i\in \A^{'}} \left(\Delta_{i} +\dfrac{320\sigma_i^2\log{( \frac{T\Delta_{i}^{2}}{K})}}{\Delta_{i}}\right).\\
\end{align*}

We obtain $(a)$ by substituting the value of $\psi$, $(b)$ from $0\leq\sigma_i^2 \leq\frac{1}{4},\forall i\in \A$ and $(c)$ from Lemma \ref{proofTheorem:Lemma:6}.\\

\noindent
\textbf{Case $(b2)$ (\textit{Optimal arm ${*}$ is eliminated by a sub-optimal arm):  }} Firstly, if conditions of Case $a$ holds then the optimal arm ${*}$ will not be eliminated in round $m=m_{*}$ or it will lead to the contradiction that $r_{i}>r^{*}$. In any round $m_{*}$, if the optimal arm ${*}$ gets eliminated then for any round from $1$ to $m_{j}$ all arms ${j}$ such that $m_{j}< m_{*}$ were eliminated according to assumption in Case $a$. Let the arms surviving till $m_{*}$ round be denoted by $\A^{'}$. This leaves any arm $a_{b}$ such that $m_{b}\geq m_{*}$ to still survive and eliminate arm ${*}$ in round $m_{*}$. Let such arms that survive ${*}$ belong to $\A^{''}$. Also maximal regret per step after eliminating ${*}$ is the maximal $\Delta_{j}$ among the remaining arms ${j}$ with $m_{j}\geq m_{*}$.  Let $m_{b}=\min\left\lbrace m|\sqrt{4\epsilon_{m}}<\frac{\Delta_{b}}{4}\right\rbrace$. Hence, the maximal regret after eliminating the arm ${*}$ is upper bounded by, 

\begin{align*}
&\sum_{m_{*}=0}^{max_{j\in \A^{'}}m_{j}}\sum_{i\in \A^{''}:m_{i}>m_{*}}\bigg(\dfrac{368 K^4}{(T^{\frac{5}{4}}\sqrt{\epsilon_{m_{*}}})} \bigg).T\max_{j\in \A^{''}:m_{j}\geq m_{*}}{\Delta}_{j}\\
%%%%%%%%%%%%%%%%%%%%%%%%%%%%
&\leq\sum_{m_{*}=0}^{max_{j\in \A^{'}}m_{j}}\sum_{i\in \A^{''}:m_{i}>m_{*}}\bigg(\dfrac{368 K^4 \sqrt{4}}{(T^{\frac{5}{4}}\sqrt{\epsilon_{m_{*}}})} \bigg).T.4\sqrt{\epsilon_{m_{*}}}\\
%%%%%%%%%%%%%%%%%%%%%%%%%%%%
&\overset{(a)}{\leq}\sum_{m_{*}=0}^{max_{j\in \A^{'}}m_{j}}\sum_{i\in \A^{''}:m_{i}>m_{*}}\bigg(\dfrac{C_2 K^4}{T^{\frac{1}{4}}\epsilon_{m_{*}}^{\frac{1}{2}-\frac{1}{2}}} \bigg)\\
%%%%%%%%%%%%%%%%%%%%%%%%%%%%
&\leq\sum_{i\in \A^{''}:m_{i}>m_{*}}\sum_{m_{*}=0}^{\min{\lbrace m_{i},m_{b}\rbrace}}\bigg(\dfrac{C_2 K^4}{T^{\frac{1}{4}}} \bigg)\\
%%%%%%%%%%%%%%%%%%%%%%%%%%%%
&\leq\sum_{i\in \A^{'}}\bigg(\dfrac{C_2 K^4}{T^{\frac{1}{4}}} \bigg)+\sum_{i\in \A^{''}\setminus \A^{'}}\bigg(\dfrac{C_2 K^4}{T^{\frac{1}{4}}} \bigg).\\
\end{align*}
Here at $(a)$, $C_2$ denotes an integer constant.



%\begin{align*}
%\sum_{m_{*}=0}^{max_{j\in \A^{'}}m_{j}}\sum_{i\in \A^{''}:m_{i}>m_{*}}\bigg(\dfrac{388 K}{(\psi  T\epsilon_{m_{*}})^{\frac{3\rho}{2}}} \bigg).T\max_{j\in \A^{''}:m_{j}\geq m_{*}}{\Delta}_{j}
%\end{align*}
%
%Again applying Lemma \ref{proofTheorem:Lemma:8} we can show that the above expression is upper bounded by 
%\begin{align*}
%\sum_{i\in \A^{'}}\dfrac{C_2^{'} K^{\frac{5}{2}}}{\sqrt{T\Delta_i}} +\sum_{i\in \A^{''}\setminus \A^{'}}\dfrac{C_2^{'} K^{\frac{5}{2}}}{\sqrt{T b}}
%\end{align*}

%%%%%%%%%%%%%%%%%%%%%%%%%%%%%%%%
%Moved to Appendix as Lemma 9
%%%%%%%%%%%%%%%%%%%%%%%%%%%%%%%%

%\begin{align*}
%&\sum_{m_{*}=0}^{max_{j\in \A^{'}}m_{j}}\sum_{i\in \A^{''}:m_{i}>m_{*}}\bigg(\dfrac{388 K}{(\psi  T\epsilon_{m_{*}})^{\frac{3\rho}{2}}} \bigg).T\max_{j\in \A^{''}:m_{j}\geq m_{*}}{\Delta}_{j}\\
%%%%%%%%%%%%%%%%%%%%%%%%%%%%%
%&\leq\sum_{m_{*}=0}^{max_{j\in \A^{'}}m_{j}}\sum_{i\in \A^{''}:m_{i}>m_{*}}\bigg(\dfrac{388 K\sqrt{4}}{(\psi  T\epsilon_{m_{*}})^{\frac{3\rho}{2}}} \bigg).T.4\sqrt{\epsilon_{m_{*}}}\\
%%%%%%%%%%%%%%%%%%%%%%%%%%%%%
%&\leq\sum_{m_{*}=0}^{max_{j\in \A^{'}}m_{j}}\sum_{i\in \A^{''}:m_{i}>m_{*}}C_2 K\bigg(\dfrac{T^{1-\frac{3\rho}{2}}}{\psi^{\frac{3\rho}{2}}\epsilon_{m_{*}}^{\frac{3\rho}{2}-\frac{1}{2}}} \bigg)\\
%%%%%%%%%%%%%%%%%%%%%%%%%%%%%
%&\leq\sum_{i\in \A^{''}:m_{i}>m_{*}}\sum_{m_{*}=0}^{\min{\lbrace m_{i},m_{b}\rbrace}}\bigg(\dfrac{C_2 K T^{1-\frac{3\rho}{2}}}{\psi^{\frac{3\rho}{2}}2^{-(\frac{3\rho}{2} -\frac{1}{2})m_{*}}} \bigg)\\
%%%%%%%%%%%%%%%%%%%%%%%%%%%%%
%&\leq\sum_{i\in \A^{'}}\bigg(\dfrac{C_2 K T^{1-\frac{3\rho}{2}}}{\psi^{\frac{3\rho}{2}}2^{-(\frac{3\rho}{2} -\frac{1}{2})m_{*}}} \bigg)+\sum_{i\in \A^{''}\setminus \A^{'}}\bigg(\dfrac{C_2 K T^{1-\frac{3\rho}{2} }}{\psi^{\frac{3\rho}{2}}2^{-(\frac{3\rho}{2} -\frac{1}{2})m_{b}}} \bigg)\\
%%%%%%%%%%%%%%%%%%%%%%%%%%%%%
%&\leq\sum_{i\in \A^{'}}\bigg(\dfrac{C_2 K T^{1-\frac{3\rho}{2}}.2^{\frac{\frac{3\rho}{2}}{2}-\frac{1}{4}}}{\psi^{\frac{3\rho}{2}}\Delta_{i}^{\frac{3\rho}{2} -\frac{1}{2}}} \bigg)+\sum_{i\in \A^{''}\setminus \A^{'}}\bigg(\dfrac{C_2 K T^{1-\frac{3\rho}{2}}.2^{\frac{\frac{3\rho}{2}}{2}-\frac{1}{4}}}{\psi^{\frac{3\rho}{2}}b^{\frac{3\rho}{2} -\frac{1}{2}}} \bigg)\\
%%%%%%%%%%%%%%%%%%%%%%%%%%%%%
%&\leq\sum_{i\in \A^{'}}\bigg(\dfrac{ C_2 K 2^{\frac{\frac{3\rho}{2}}{2}+\frac{19}{4}}.T^{1-\frac{3\rho}{2} } }{\psi^{\rho}\Delta_{i}^{2\frac{3\rho}{2} -1}} \bigg)+\sum_{i\in \A^{''}\setminus \A^{'}}\bigg(\dfrac{C_2 K 2^{\frac{\frac{3\rho}{2}}{2}+\frac{19}{4}}.T^{1-\frac{3\rho}{2}} }{\psi^{\frac{3\rho}{2} }b^{2\frac{3\rho}{2}-1}} \bigg)\\
%%%%%%%%%%%%%%%%%%%%%%%%%%%%%
%&\overset{(a)}{\leq}\sum_{i\in \A^{'}}\bigg(\dfrac{C_2^{'} K .T^{1-\frac{3}{4}}}{(\frac{T}{K^2})^{\frac{3}{4}}\Delta_{i}^{2.\frac{3}{4} -1}} \bigg)+\sum_{i\in \A^{''}\setminus \A^{'}}\bigg(\dfrac{C_2^{'} K T^{1-\frac{3}{4}}}{(\frac{T}{K^2})^{\frac{3}{4}}b^{2.\frac{3}{4}-1}} \bigg)\\
%%%%%%%%%%%%%%%%%%%%%%%%%%%%%
%&\leq\sum_{i\in \A^{'}}\dfrac{C_2^{'} K^{\frac{5}{2}}}{\sqrt{T\Delta_i}} +\sum_{i\in \A^{''}\setminus \A^{'}}\dfrac{C_2^{'} K^{\frac{5}{2}}}{\sqrt{T b}}
%%%%%%%%%%%%%%%%%%%%%%%%%%%%%
%\end{align*}
%In the above simplification, $(a)$ is obtained by substituting the values of $\psi$ and $\rho$.

Finally, summing up the regrets in \textbf{Case a} and \textbf{Case b}, the total regret is given by
\begin{align*}
\E [R_{T}] \leq &\sum\limits_{i\in \A :\Delta_{i} > b}\bigg\lbrace \dfrac{C_0 K^{4}}{T^{\frac{1}{4}}} + \bigg(\Delta_{i}+\dfrac{320\sigma_i^2\log{(\frac{T\Delta_{i}^{2}}{K})}}{\Delta_{i}}\bigg)\bigg \rbrace\\ 
  & +\sum\limits_{i\in \A :0 < \Delta_{i}\leq b} \dfrac{C_2 K^{4}}{T^{\frac{1}{4}}} + \max_{i\in \A :0 < \Delta_{i}\leq b}\Delta_{i}T
\end{align*}

where $C_0, C_1, C_2$ are integer constants and $C_0 = C_1 + C_2$.
\end{customproof}




\section{Experimental Section}
\label{sec:expt}
In this section we conduct an extensive empirical evaluation of EUCBV against several other popular bandit algorithms.  We use cumulative regret as the metric of comparison. We implement the following algorithms:  KL-UCB \citep{garivier2011kl}, DMED \citep{honda2010asymptotically}, MOSS \citep{audibert2009minimax}, UCB1 \citep{auer2002finite}, UCB-Improved \citep{auer2010ucb}, Median Elimination \citep{even2006action}, Thompson Sampling(TS) \citep{agrawal2011analysis}, OCUCB \citep{lattimore2015optimally}, Bayes-UCB (BU) \citep{kaufmann2012bayesian} and UCB-V \citep{audibert2009exploration}\footnote{The implementation for KL-UCB, Bayes-UCB and DMED were taken from \citet{CapGarKau12}}. The parameters of EUCBV algorithm for all the experiments are set as follows: $\psi=\frac{T}{K^2}$ and $\rho =0.5$ (as in Corollary \ref{Result:Corollary:2}).

\begin{figure}[!h]
    \centering
    \begin{tabular}{cc}
    \setlength{\tabcolsep}{0.1pt}
    \subfigure[0.25\textwidth][Expt-$1$: $20$ Bernoulli-distributed arms with $r_{i_{{i}\neq {*}}}=0.07$ and $r^{*}=0.1$]
    {
    		\pgfplotsset{
		tick label style={font=\Large},
		label style={font=\Large},
		legend style={font=\Large},
		ylabel style={yshift=32pt},
		%legend style={legendshift=32pt},
		}
        \begin{tikzpicture}[scale=0.5]
      	\begin{axis}[
		xlabel={timestep},
		ylabel={Cumulative Regret},
		grid=major,
        %clip mode=individual,grid,grid style={gray!30},
        clip=true,
        %clip mode=individual,grid,grid style={gray!30},
  		legend style={at={(0.5,1.5)},anchor=north, legend columns=3} ]
      	% UCB
		\addplot table{results/NewExpt/Expt1/UCBV01_comp_subsampled.txt};
		%\addplot table{results/NewExpt/Expt1/EclUCB01_1_comp_subsampled.txt};
		\addplot table{results/NewExpt/Expt1/EUCBV01_comp_subsampled.txt};
		\addplot table{results/NewExpt/Expt1/KLUCB01_comp_subsampled.txt};
		\addplot table{results/NewExpt/Expt1/MOSS01_comp_subsampled.txt};
		\addplot table{results/NewExpt/Expt1/DMED01_comp_subsampled.txt};
		\addplot table{results/NewExpt/Expt1/UCB01_comp_subsampled.txt};
		\addplot table{results/NewExpt/Expt1/TS02_comp_subsampled.txt};
		\addplot table{results/NewExpt/Expt1/OCUCB01_comp_subsampled.txt};
		\addplot table{results/NewExpt/Expt1/BU01_comp_subsampled.txt};
      	\legend{UCB-V,EUCBV,KL-UCB,MOSS,DMED,UCB1,TS,OCUCB,BU}      	
      	\end{axis}
      	\end{tikzpicture}
  		\label{fig:1}
    }
    &
    \subfigure[0.25\textwidth][Expt-$2$: $100$ Gaussian-distributed arms with $r_{i_{{i}\neq {*}:1-33}}=0.1$, $r_{i_{{i}\neq {*}:34-99}}=0.6$, $r^{*}_{i=100}=0.9$ and $\sigma_{i=1:100}^{2} = 0.3$]
    {
    		\pgfplotsset{
		tick label style={font=\Large},
		label style={font=\Large},
		legend style={font=\Large},
		}
        \begin{tikzpicture}[scale=0.5]
        \begin{axis}[
		xlabel={timestep},
		ylabel={Cumulative Regret},
        %clip mode=individual,grid,grid style={gray!30},
       	grid=major,
       	clip=true,
  		legend style={at={(0.5,1.5)},anchor=north, legend columns=3} ]
      	% UCB
        \addplot table{results/NewExpt/Expt2_2/UCB01_comp_subsampled.txt};
		%\addplot table{results/NewExpt/Expt2_2/clUCB01_comp_subsampled.txt};
		\addplot table{results/NewExpt/Expt2_2/EUCBV01_comp_subsampled.txt};
		\addplot table{results/NewExpt/Expt2_2/MOSS01_comp_subsampled.txt};
		\addplot table{results/NewExpt/Expt2_2/OCUCB01_comp_subsampled.txt};
		\addplot table{results/NewExpt/Expt2_2/MedElim_comp_subsampled.txt};
		\addplot table{results/NewExpt/Expt2_2/UCBR01_comp_subsampled.txt};
		\addplot table{results/NewExpt/Expt2_2/UCBV01_comp_subsampled.txt};
		%\legend{UCB1,ClusUCB,Med-Elim,MOSS,OCUCB,EClusUCB,UCB-Imp}
		\legend{UCB1,EUCBV,MOSS,OCUCB,Med-Elim,UCB-Imp,UCBV}
      	\end{axis}
      	\end{tikzpicture}
   		\label{fig:2}
    }
    \end{tabular}
    \caption{Cumulative regret for various bandit algorithms on two stochastic K-armed bandit environments. }
    \label{fig:karmed}
    \vspace*{-1em}
\end{figure}
% For the purpose of performance comparison


\textbf{First experiment:} This experiment is conducted to observe the performance of EUCBV over a short horizon. The horizon $T$ is set to $60000$. These type of cases are frequently encountered in web-advertising domain. The testbed comprises of $20$ Bernoulli distributed arms with expected rewards of the arms as $r_{i_{{i}\neq {*}}}=0.07$ and $r^{*}=0.1$. The regret is averaged over $100$ independent runs and is shown in Figure \ref{fig:1}. EUCBV, MOSS, UCB1, UCB-V, KL-UCB, TS, BU and DMED are run in this experimental setup. Here not only we observe that EUCBV performs better than all the non-variance based  algorithms like MOSS, OCUCB, UCB-Improved and UCB1 but it also outperforms UCBV because of the choice of the exploration parameters. Because of the small gaps and short horizon $T$, we do not implement UCB-Improved and Median Elimination on this test-case. 

\textbf{Second experiment:} This experiment is conducted on a large horizon and over a large set of arms.    The horizon $T$ is set for a large duration of $2\times 10^{5}$. This testbed comprises of $100$ arms involving Gaussian reward distributions with expected rewards of the arms $r_{1:33}=0.1$, $r_{34:99}=0.6$, $r^{*}_{100}=0.9$ and variance set at $\sigma_{i}^{2} = 0.3,\forall i\in \A$. The regret is averaged over $100$ independent runs and is shown in Figure \ref{fig:2}. From the results in Figure \ref{fig:2}, we observe that EUCBV outperforms all the non-variance based algorithms MOSS, OCUCB, UCB1, UCB-Improved and Median-Elimination($\epsilon=0.1,\delta=0.1$). 
%Also the performance of UCB-Improved is poor in comparison to other algorithms, which is probably because of pulls wasted in initial exploration whereas EUCBV with the choice of $\psi$ and $\rho$ performs much better.

\begin{figure}[!h]
    \centering
    \begin{tabular}{cc}
    \subfigure[0.25\textwidth][Expt-$3$: $20$ to $100$ Bernoulli-distributed arms with $r_{i_{{i}\neq {*}}}=0.05$ and $r^{*}=0.1$]
    {
    	\pgfplotsset{
		tick label style={font=\Large},
		label style={font=\Large},
		legend style={font=\Large},
		ylabel style={yshift=32pt},
		}
        \begin{tikzpicture}[scale=0.45]
        \begin{axis}[
		xlabel={Arms},
		ylabel={Cumulative Regret},
        %clip mode=individual,grid,grid style={gray!30},
		grid=major,
		clip=true,
  		legend style={at={(0.5,1.3)},anchor=north, legend columns=3} ]
        % UCB
		\addplot table{results/NewExpt/Expt3/plotFinalMOSS20_100.txt};
		\addplot table{results/NewExpt/Expt3/plotFinalEUCBV20_100.txt};
		\addplot table{results/NewExpt/Expt3/plotFinalOCUCB20_100.txt};
      	\legend{MOSS,EUCBV,OCUCB}
      	\end{axis}
        \end{tikzpicture}
        \label{fig:3}
    }
    &
    \subfigure[0.25\textwidth][Expt-$4$ (Advanced Setting): $r_{1:33}=0.4$, $r_{34:99}=0.6$, $r^{*}_{100}=0.9$; $\sigma_{1:33}^{2}=0.2$, $\sigma_{34:99}^{2}=0.1$,  $\sigma_{100}^{2}=0.4$]
    {
    		\pgfplotsset{
		tick label style={font=\Large},
		label style={font=\Large},
		legend style={font=\Large},
		ylabel style={yshift=32pt},
		}
        \begin{tikzpicture}[scale=0.45]
      	\begin{axis}[
		ylabel={Cumulative Regret},
		xlabel={timestep},
		grid=major,
        %clip mode=individual,grid,grid style={gray!30},
        clip=true,
        %clip mode=individual,grid,grid style={gray!30},
  		legend style={at={(0.5,1.3)},anchor=north, legend columns=3} ]
      	% UCB
		\addplot table{results/NewExpt/Expt4/UCBV01_comp_subsampled.txt};
		\addplot table{results/NewExpt/Expt4/EUCBV01_comp_subsampled.txt};
		\addplot table{results/NewExpt/Expt4/MOSS01_comp_subsampled.txt};
		\addplot table{results/NewExpt/Expt4/OCUCB01_comp_subsampled.txt};
      	%\legend{EClusUCBA,AClusUCBA,EClusUCB,AClusUCB} 
      	\legend{UCBV,EUCBV,MOSS,OCUCB} 
      	\end{axis}
      	\end{tikzpicture}
  		\label{fig:4}
    }
	\end{tabular}
	\label{fig:furtherExpt1}
    \caption{Further Experiments with EUCBV}
    \vspace*{-1em}
\end{figure}
%\vspace*{-0.5em}
\textbf{Third experiment:} This experiment is conducted to show the stability and performance of EUCBV over a very large horizon and over a large number of arms. This testbed comprises of $20-100$ (interval of $10$) arms with Bernoulli reward distributions, where the expected rewards of the arms are $r_{i_{{i}\neq {*}}}=0.05$ and $r^{*}=0.1$. For each of these testbeds of $20-100$ arms, we report the cumulative regret averaged over $100$ independent runs. The horizon is set at $T=10^{5} + K_{20:100}^{3}$ timesteps. Please note that algorithms like Thompson Sampling or Bayes-UCB are too slow to be run for such large arms (see \citet{lattimore2015optimally}) and  over such large horizon. We report the performance of MOSS, OCUCB and EUCBV only over this uniform gap setup. From the results in Figure \ref{fig:3}, it is evident that the growth of regret for EUCBV  is much lower than that of OCUCB and MOSS. 

%This also corroborates the finding of \citet{lattimore2015optimally} which states that MOSS breaks down only when the number of arms are exceptionally large or the horizon is unreasonably high and gaps are very small. We consistently see that in uniform gap testcases EUCBV outperforms OCUCB.

\textbf{Fourth experiment:} This experiment is conducted on 100 Gaussian distributed arms such that expected rewards of the arms $r_{1:33}=0.4$, $r_{34:99}=0.6$, $r^{*}_{100}=0.9$ and the variance is set as $\sigma_{1:33}^{2}=0.2$, $\sigma_{34:99}^{2}=0.1$,  $\sigma_{100}^{2}=0.4$ and $T=3\times 10^5$. This experiment is conducted to show that in certain environments, when the variance of the optimal arm is higher than other sub-optimal arms, then EUCBV performs exceptionally well. We refer to this setup as Advanced Setting because here the chosen variance values are such that only variance-aware algorithms will perform well because the variance of the optimal arm is chosen to be higher than the other arms and so algorithms that are not variance-aware will spent a significant amount of pulls trying to find the optimal arm. The result is shown in Figure \ref{fig:4}. Predictably EUCBV, which allocates pulls proportional to the variance of the arms, outperforms MOSS and OCUCB. Note that EUCBV by virtue of its aggressive exploration parameters outperforms UCBV even though UCBV is a variance based algorithm.





\section{Conclusion and Future Works}
\label{sec:conc}
In this paper, we studied the EUCBV algorithm which takes into account the empirical variance of the arms and employs aggressive exploration parameters to eliminate sub-optimal arms. Our theoretical analysis conclusively established that EUCBV enjoys an order-optimal gap-independent regret bound of $O\left(\sqrt{KT}\right)$. Empirically we showed that EUCBV performs superbly across diverse experimental settings and outperforms most of the bandit algorithms in stochastic  MAB setup. Our experiments showed that EUCBV is extremely stable for larger horizons and performs superbly  across different types of distributions. One future work is to remove the constraint of $T\geq K^{2.4}$ required for EUCBV to reach the order optimal regret bound. Another future direction is to come up with an anytime version of EUCBV. 


\clearpage
\newpage

\bibliography{biblio}
\bibliographystyle{apalike}

\end{document}


\section{Theoretical Analysis of EUCBV}
\begin{frame}
\frametitle{Expected Regret of EUCBV}
%
%\begin{theorem}
%For $T\geq K^{2.4}$, $\rho=\frac{1}{2}$ and $\psi=\frac{T}{K^2}$, the regret $R_T$ for EUCBV satisfies
%\begin{align*}
%\E [R_{T}] \leq &\sum\limits_{i\in \A :\Delta_{i} > b}\bigg\lbrace \dfrac{C_0 K^{4}}{T^{\frac{1}{4}}} + \bigg(\Delta_{i}+\dfrac{320\sigma_i^2\log{(\frac{T\Delta_{i}^{2}}{K})}}{\Delta_{i}}\bigg)\bigg \rbrace\\ 
%  & +\sum\limits_{i\in \A :0 < \Delta_{i}\leq b} \dfrac{C_2 K^{4}}{T^{\frac{1}{4}}} + \max_{i\in \A :0 < \Delta_{i}\leq b}\Delta_{i}T.
%\end{align*}
%
%for all $b\geq\sqrt{\frac{e}{T}}$ and $C_0, C_2$ are integer constants. 
%
%\end{theorem}

\begin{corollary}[\textbf{\textit{Gap-Independent Bound}}]
\label{Result:Corollary:1}
The regret of EUCBV is upper bounded by the following gap-independent expression:
\begin{align*}
	\E[R_{T}]\leq  \frac{C_3 K^5}{T^{\frac{1}{4}}} + 80\sqrt{KT}.
\end{align*}	
\end{corollary}


\begin{table}[b]
%\caption{AugUCB vs.\ State of the art}
\label{tab:comp-bds}
\begin{center}
\begin{tabular}{|p{1.5cm}|p{3.1cm}|p{3.1cm}|p{1.0cm}|}
% \toprule
\hline
Algorithm  & GD Bound & GI Bound & Var \\
% \midrule
\hline
%\hline
EUCBV      &$O\left( \frac{K\sigma_{\max}^{2}\log (\frac{T\Delta^2}{K})}{\Delta}\right)$ & $O\left(\sqrt{KT}\right)$ & Yes\\
%\hline
\hline
UCBV		&$O\left( \frac{K\sigma_{\max}^{2}\log T}{\Delta} \right)$ & $O\left(\sqrt{KT\log T}\right)$ & Yes\\
%\midrule
%\hline
\hline
MOSS         &$O\left( \frac{K^2\log (T\Delta^2 /K)}{\Delta}\right)$ & $O\left(\sqrt{KT}\right)$ & No\\
% \midrule
%\hline
\hline
OCUCB		&$O\left( \frac{K\log (T/ H_{i})}{\Delta}\right)$ &  $O\left(\sqrt{KT}\right)$ & No\\
%\midrule
\hline

%\bottomrule
\end{tabular}
\end{center}
\end{table}


%\begin{table}[tbp]
%%\caption{Regret upper bound of different algorithms}
%\label{tab:comp-bds}
%\begin{center}
%\begin{tabular}{p{3em}p{9em}p{7em}}
%\toprule
%Algorithm  &   \hspace*{1mm}Gap-Dependent & Gap-Independent \\
%\hline
%EUCBV		& $O\left( \dfrac{K\sigma_{\max}^{2}\log (\frac{T\Delta^2}{K})}{\Delta}\right)$ & $O\left(\sqrt{KT}\right)$\\
%%UCB1        & $O\left( \dfrac{K\log T}{\Delta} \right)$ & $O\left(\sqrt{KT\log T}\right)$ \\%\midrule
%UCBV        & $O\left( \dfrac{K\sigma_{\max}^{2}\log T}{\Delta} \right)$ & $O\left(\sqrt{KT\log T}\right)$ \\
%%UCB-Imp 		& $O\left( \dfrac{K\log (T\Delta^2)}{\Delta} \right)$ & $O\left(\sqrt{KT\log K}\right)$ \\%\midrule
%MOSS	     	& $O\left( \dfrac{K^2\log (T\Delta^2 /K)}{\Delta}\right)$ & $O\left(\sqrt{KT}\right)$\\%\midrule
%OCUCB     	& $O\left( \dfrac{K\log (T/ H_{i})}{\Delta}\right)$ & $O\left(\sqrt{KT}\right)$\\\midrule
%\end{tabular}
%\end{center}
%\vspace*{-2em}
%\end{table}



\end{frame}

\section{Experiments in SMAB}
\begin{frame}
\frametitle{Experiments in SMAB}
\begin{figure}[tbp]
    \centering
    \begin{tabular}{cc}
    \subfigure[0.25\textwidth][Expt-$1$: Failure of TS]
    {
    		\pgfplotsset{
		tick label style={font=\Large},
		label style={font=\Large},
		legend style={font=\Large},
		ylabel style={yshift=9pt},
		}
        \begin{tikzpicture}[scale=0.5]
      	\begin{axis}[
		ylabel={Cumulative Regret},
		xlabel={timestep},
		grid=major,
        %clip mode=individual,grid,grid style={gray!30},
        clip=true,
        %clip mode=individual,grid,grid style={gray!30},
  		legend style={at={(0.5,1.2)},anchor=north, legend columns=3} ]
      	% UCB
		\addplot table{results1/NewExpt/Expt3/UCBV01_comp_subsampled.txt};
		\addplot table{results1/NewExpt/Expt3/EUCBV01_comp_subsampled.txt};
		\addplot table{results1/NewExpt/Expt3/MOSS01_comp_subsampled.txt};
		\addplot table{results1/NewExpt/Expt3/TS01_comp_subsampled.txt};
		\addplot table{results1/NewExpt/Expt3/OCUCB01_comp_subsampled.txt};
		\addplot table{results1/NewExpt/Expt3/BU01_comp_subsampled.txt};
      	\legend{UCBV,EUCBV,MOSS,TS-G,OCUCB,BU-G} 
      	\end{axis}
      	\end{tikzpicture}
  		\label{fig:3}
    }
    &
    \subfigure[0.25\textwidth][Expt-$2$: $3$ Group Variance]
    %with $r_{i_{{i}\neq {*}}}=0.05$ and $r^{*}=0.1$
    {
    	\pgfplotsset{
		tick label style={font=\Large},
		label style={font=\Large},
		legend style={font=\Large},
		ylabel style={yshift=9pt},
		}
        \begin{tikzpicture}[scale=0.5]
        \begin{axis}[
		xlabel={timestep},
		ylabel={Cumulative Regret},
        %clip mode=individual,grid,grid style={gray!30},
		grid=major,
		clip=true,
  		legend style={at={(0.5,1.2)},anchor=north, legend columns=3} ]
        % UCB
		\addplot table{results1/NewExpt/Expt41/UCBV01_comp_subsampled.txt};
		\addplot table{results1/NewExpt/Expt41/EUCBV01_comp_subsampled.txt};
		\addplot table{results1/NewExpt/Expt41/MOSS01_comp_subsampled.txt};
		\addplot table{results1/NewExpt/Expt41/TS01_comp_subsampled.txt};
		\addplot table{results1/NewExpt/Expt41/OCUCB01_comp_subsampled.txt};
		\addplot table{results1/NewExpt/Expt41/BU01_comp_subsampled.txt};
      	\legend{UCBV,EUCBV,MOSS,TS-G,OCUCB,BU-G} 
      	\end{axis}
        \end{tikzpicture}
        \label{fig:4}
    }
	\end{tabular}
	\label{fig:furtherExpt1}
    %\caption{Experiments with EUCBV}
\end{figure}
\end{frame}

\section{Conclusions}
\begin{frame}
\frametitle{Conclusions}
\begin{itemize}
\item<1-> We proposed the EUCBV algorithm for the SMAB setting which uses variance and mean estimation along with arm elimination to find the optimal arm.
\item<2-> Theoretically, EUCBV achieves an order-optimal regret guarantees, but further studies are required to reduce the constants.
\item<3-> A more detailed analysis of the non-uniform arm selection and parameter selection is also required for EUCBV.
\item<4-> \textbf{Acknowledgement: } We thank Google India for granting us with a generous travel grant to present our work in AAAI 2018.
\end{itemize}
\end{frame}


%\section{References}
%\begin{frame}[allowframebreaks]
%\frametitle{References}
%%\bibliographystyle{named} 
%\bibliographystyle{plainnat} 
%%\bibliographystyle{dinat}
%\bibliography{ijcai17}
%\end{frame}


%------------------------------------------------

\begin{frame}
\Huge{\centerline{Thank You}}
\end{frame}

%----------------------------------------------------------------------------------------

\end{document} 